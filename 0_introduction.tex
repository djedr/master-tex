\chapter{Introduction}\label{chap:intro}
%\js{Rozdziały zwyczajowo numeruje się od 1, nie od 0.}

\section{Scope}
This research spans a fairly broad area of knowledge, connecting -- in order of importance -- programming language design, web technologies, web application design and development as well as computer game development.

The main focus of this thesis is designing a programming language, which can have multiple deeply integrated editable representations.

I present a way to combine features of visual languages and text-based languages in an integrated development environment, which lets the programmer work with both representations in parallel or intertwine them in various ways.

A proof-of-concept interpreter and development environment for the language is implemented using web technologies.

Practical demonstration of the capabilities of the implementation is presented by writing a Pac-Man clone in the designed language\footnote{The term ``Pac-Man-complete'' in the title of this thesis refers to a somewhat humorous description used by the Idris' programming language\cite{idris} author, Edwin Brady\cite{pacman_complete, type_theory_podcast}, to describe the language. In the context of this dissertation it means that the designed programming language provides enough features to allow one to write a clone of the classic Pac-Man game in it.}. This also provides a reference for assessing the performance of the implementation.

\section{Choice of subject}
The choice of this particular subject stems from my deep personal interest in programming language design. This research is an opportunity for me to create a project that demonstrates various ideas in this area that I developed over time and to explore and refine them further.

\clearpage
\section{Related work}
With this research, I intend to explore certain aspects of programming language design as well as further the growth of visual programming languages, proposing a solution that improves over any existing comparable technology in terms of simplicity, expressive power of the language and usability.

The first and main part of this research is concerned with designing a programming language, which becomes the basis for the second part. This language, Dual, is mostly based on one of the oldest \acrshort{pl}\footnote{For the sake of terseness I will sometimes use the acronyms \acrshort{pl} and \acrshort{vpl} to abbreviate ``programming language'' and ``visual programming language''.} families, the Lisp\cite{lisp_draft, lisp_wikipedia} family. Lisp and its various dialects are consistently regarded as one of the most expressive PLs\cite{lisp_expressive, lisp_powerful}, despite having a very simple syntactic and semantic core\cite{lisp_simple}. Dual is also influenced by many modern PLs, such as JavaScript (as defined by the ECMAScript\cite{ecmascript}), which is the implementation language of its interpreter. Similarly to Lisp, Dual has a minimal syntax, although with some modifications and improvements (described in Chapter \ref{chap:lang}). 

The second part of this research builds on top of theoretical\cite{visual_languages} as well as practical\cite{snapshots} achievements in the field of \acrlong{vpl}s (\acrshort{vpl}s), focusing on examining the latter. VPLs can be classified in various ways: \cite[Section~VPL-II.B]{visual_languages}, \cite[Section~Types~of~VPLs]{vpl_maturity}, \cite[Section~Definition]{vpl_wikipedia}. I introduce my own classification by examining lanugages enumerated in \cite{snapshots}. Two major categories\footnote{By amount of languages that fall into each.} that emerge from this classification are ``line-connected block-based'' and ``snap-together block-based'' VPLs. I design and implement a visual representation for Dual, which combines features characteristic to both of these categories. The development environment that is built for the language provides the ability to edit the text and the visual representation in parallel, with the changes made to one visible in the other immediately.

Visual languages are not especially popular compared to text-based languages. But recently they have been gaining more popularity, particularly in game development. Chief example and the main cause of this is Unreal Engine, the highly popular and mainstream\cite{unreal_list, unreal_guinness} game engine, which in the latest version introduced a visual programming language\cite{blueprint} as its primary scripting language. In fact this is the only scripting language that the engine supports, having dropped the UnrealScript language\cite{unreal_script} included in the previous versions. This visual programming language will be compared to Dual to highlight its advantages.

The Dual language, both its representations and its environment -- I will further use the terms ``Dual system'' or simply ``system'' to refer to these as a whole -- are built entirely on top of the open web platform\cite{open_web_platform}, which is ubiquitous. This gives the system great portability and makes the difficulty for the potential user to start working with it minimal. This is how the usability mentioned in the first paragraph of this section is defined.

\section{Goals}
In line with the above, the purpose of this work is to introduce innovation as well as show a practical application of the developed solution. The concrete goals are:
\begin{itemize}
	%\item To explore and establish directions where innovation is possible in programming language design and implementation.
	\item To design a programming language, which meets the criteria of expressiveness and usability outlined in the previous section.
	\item To provide a \textit{general} design of the development environment for the language. This design must include an editor for a visual representation of the language, which must be directly mappable to the text form. Both forms must be designed to be used interchangeably.
    \item To implement a prototype of the designed environment, including an interpreter for the language, a text editor and a visual editor that conform to the main design requirements.
    \item To evaluate the practical usability and performance of the prototype by creating a clone of Pac-Man and examine the process as well as the results.
    \item To present possible ways of improving existing visual language systems.
\end{itemize}

\section{Structure}
This thesis is structured as follows:

Chapter \ref{chap:intro} is this introduction.

Chapter \ref{chap:background} briefly describes technologies and tools used in developing any software described here as well as discusses the essential elements of the theoretical framework upon which the language was built.

Chapter \ref{chap:lang} describes the design of the Dual programming language: its syntax, semantics, primitives, core functions and values. It also elaborates on programming language design in general.

Chapter \ref{chap:editor} talks about the design of the language's visual representation and its development environment. 

Chapter \ref{chap:impl} describes the prototype implementation based on the designs. This includes the language and its intrepreter as well as the environment.

Chapter \ref{chap:case} contains a case study of a more-than-trivial application developed with Dual: a Pac-Man clone. Performance of the prototype implementation is assessed and possible adjustments and improvements are discussed.

Chapter \ref{chap:comp} compares Dual to existing visual programming languages.

Chapter \ref{chap:summary} summarizes and concludes.

Appendix \ref{app:dvd} describes the contents of the DVD attached to this thesis and provides a short instruction on running the prototype.

Appendix \ref{app:design} contains additional design ideas that may be implemented in the future.
