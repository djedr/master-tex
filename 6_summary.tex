\chapter{Summary and conclusions}\label{chap:summary}
I intend to continue my research with the goal of creating a modern real-world
programming language useful for a specific range of tasks

I believe that there is room for a small and simple scripting language, which fuses Lisps, JavaScript's, Lua's best and most powerful features. Adding to that full support for visual programming designed to fix as much as possible of the obvious shortcomings of current visual languages as well as introducing innovation always providing direct mapping and fallback to text representation  

If support for visual programming is good and innovative
It has a chance to draw attention of designers and 

The visual programming feature could draw the attention of non-programmers, designers and programmers, who work by means of prototyping and exploratory programming

Perhaps the above combination of features
If well designed and executed
Might be enough to outweigh \cite{pl_checklist}
Create a language, which does not tick too many boxes
Not ticking too many boxes

\js{Ten akapit i kilka następnych brzmią jak z podsumowania.  Sugeruję przenieść
    albo do rozdziału z podsumowaniem, ewentualnie na końcu rozdziału zrobić
    podrozdział z wnioskami.  Na pewno nie powinno być tego we wstępie zanim
    cokolwiek zostanie zaprezentowane.}

While implementing this project I learned that programming language design is a
tremendous task, especially if the language being designed is intended to be of
real-world use. Designing and implementing such a language absolutely from
scratch, while introducing useful innovation cannot be done within the time
limits of research for a thesis, unless perhaps by an experienced language
designer. But such experience has to be gained somehow and this is an excellent
opportunity.

The character of this research project is exploratory, although I intend to
further develop ideas described here and continue my research, which will, as I
hope, eventually result in creation of an innovative and useful language ready
for real-world use.

That aside, I believe that at least some of the ideas described here are -- in a
varying degree -- innovative and worth exploring further.

Even though the language presented in this thesis is complete in the sense of
being able to implement any algorithm and non-trivial applications, as
exemplified by the Pac-Man clone, it is by no means a complete design. It should
be viewed as a snapshot from a continuous design process that is intended to
progress in the future.\js{Wydaje mi się, że ten akapit równiez pownien być w podsumowaniu}