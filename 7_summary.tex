\chapter{Summary and conclusions}\label{chap:summary}
I intend to continue my research with the goal of creating a modern real-world
programming language useful for a specific range of tasks

I believe that there is room for a small and simple scripting language, which fuses Lisps, JavaScript's, Lua's best and most powerful features. Adding to that full support for visual programming designed to fix as much as possible of the obvious shortcomings of current visual languages as well as introducing innovation always providing direct mapping and fallback to text representation  

If support for visual programming is good and innovative
It has a chance to draw attention of designers and 

The visual programming feature could draw the attention of non-programmers, designers and programmers, who work by means of prototyping and exploratory programming

Perhaps the above combination of features
If well designed and executed
Might be enough to outweigh \cite{pl_checklist}
%Create a language, which does not tick too many boxes
%Not ticking too many boxes

\js{Ten akapit i kilka następnych brzmią jak z podsumowania.  Sugeruję przenieść
    albo do rozdziału z podsumowaniem, ewentualnie na końcu rozdziału zrobić
    podrozdział z wnioskami.  Na pewno nie powinno być tego we wstępie zanim
    cokolwiek zostanie zaprezentowane.}

While implementing this project I learned that programming language design is a
tremendous task, especially if the language being designed is intended to be of
real-world use. Designing and implementing such a language absolutely from
scratch, while introducing useful innovation cannot be done within the time
limits of research for a thesis, unless perhaps by an experienced language
designer. But such experience has to be gained somehow and this is an excellent
opportunity.

The character of this research project is exploratory, although I intend to
further develop ideas described here and continue my research, which will, as I
hope, eventually result in creation of an innovative and useful language ready
for real-world use.

That aside, I believe that at least some of the ideas described here are -- in a
varying degree -- innovative and worth exploring further.

Even though the language presented in this thesis is complete in the sense of
being able to implement any algorithm and non-trivial applications, as
exemplified by the Pac-Man clone, it is by no means a complete design. It should
be viewed as a snapshot from a continuous design process that is intended to
progress in the future.\js{Wydaje mi się, że ten akapit równiez pownien być w podsumowaniu}

For these reasons the created environment is by no means complete and ready for
use in developing complex applications. The degree of completeness of the
project is reflected in:
\begin{itemize}
    \item The core language, which is sufficiently expressive to implement any
    algorithm, i.e. Turing-complete\footnote{In the same sense as JavaScript or C. No existing language is Turing-complete in the absolute sense, because of physical hardware limitations.}. A simple Brainfuck interpreter is included as an example program (see Appendix \ref{app:plyta}) to demonstrate this\cite{bf_turing_complete}.   
    \item Implementation of several interesting additional features of the core
    language, which are described in this chapter \js{wymienić jakich}
    \item The ablility of the language to implement also non-trivial
    applications. This is demonstrated by implementing a clone of Pac-Man,
    described in Chapter \ref{chap:case}
    \item The direct correspondence of the visual and text representations, with
    the possibility of parallel dynamic editing; albeit the visual editing
    part of the editor includes only basic features and is not optimized in
    terms of performance; details are described in Chapter \ref{chap:editor}
    \item Implementation of the prototype of the language's development
    environment on top of the web platform, which includes a server-side and a
    client-side part. It runs locally on the user's machine, but is designed
    to be easily deployed as an online web application
    \item The editor has several useful features, such as basic support for text
    editing built on top of the CodeMirror JavaScript component with custom
    syntax highlighting and integration with the editor. The text editing
    component is integrated with the visual editing component, so that
    navigation or changes to each representation are tracked and visible to
    the user [[]]
    \item [[]]
\end{itemize}

Language
    extensible
    expressive


Visual programming systems that connected text and visual representations to were developed in the past\cite{snapshots}. But none of them seems to have a similar degree of integration as the system presented here.

While doing this research for this thesis I found a VPL project named Luna, that may be built around a similar idea of highly integrated text and visual representations. The project's website claims:
\cite{luna_website}:
\begin{quote}
Luna is the world’s first programming language featuring two exchangeable representations: textual and visual, which you can freely switch at any time.
\end{quote}

However the project is in ``private alpha'' stage -- it is not publicly available as of July 2016. Nonetheless this perhaps illustrated that the ideas presented in this research are a step on a path to better future VPL systems. 