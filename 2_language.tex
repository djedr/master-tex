\chapter{Dual programming language design}\label{chap:lang}
This chapter describes the text representation of Dual, which is the basis for the executable representation for the interpreter (the syntax tree) and for the block-based visual representation discussed in the following chapters.

The evolution of programming languages is a gradual process. And so is the process of designing a single language. The approach that I found effective was iterative refinement, addition, testing, and sometimes subtraction of features. In practice this translates to intermediate designs and implementations being rearranged into new forms, with some discarded. I did not arrive at something that I could call the final form of the language, so a lot of the features described here are subject to change and improvement. I intend to work on this project further beyond the scope of this thesis.

The language was not originally intended to be a Lisp-like language or clone thereof, but throughout the research I ended up learning a lot about Lisp, sometimes by reinventing parts of this language. A somewhat philosophical interpretation of this would be that Lisp is built on fundamental principles that are (re)discoverable rather than invented.

In this and the following chapters I cover a lot of ``design surface'', only delving deep into some features that are relevant to core ideas that I wanted to convey in this thesis.

\section{Grammar and syntactic features}
Among the main design goals for the prototype of the language were simplicity
and clarity. I wanted a language that is easy to parse and transform to a
different representation. This restriction suggests that the syntax should be as
minimal as possible.  

I choose Lisp's syntax as the starting point. It is indeed almost as simple as one could imagine. But because of its almost complete uniformity it is often criticized. Some of the major criticisms are:
\begin{itemize}
    \item In general it is hard to teach, because complex code gets easily confusing\cite{wadler_critique}.
    \item The more nested the syntax tree, the harder it is to keep track of and
      balance parentheses; there tends to be a lot of closing parentheses next
      to each other in the source\cite{c2_parentheses}.
\end{itemize}

I made a few simple adjustments to the syntax in order to address these concerns, at least to some degree. These modifications do not significantly increase the complexity of a parser, but may considerably improve the syntax in terms of ease of use and readability for a human.

\subsection{Basic syntax}
Below I present the definition of Dual's grammar in left-recursion-free \acrlong{bnf}. It is included here only for the sake of formality. I believe that for such a simple grammar BNF introduces more noise and is unnecessarily more complex than a textual description, possibly with the help of regular expressions or simply verbatim parser source code. For these reasons any extensions to this basic grammar will later on be introduced in these ways.

This is the \acrshort{bnf} definition, a bit verbose for clarity:
\begin{lstlisting}
<expression> ::= <word> | <call>
<call> ::= <operator> <argument-list>
<operator> ::= <word> <argument-lists>
<argument-list> ::= "[" <arguments> "]"
<word> ::= /[^\s\[\]]+/
<argument-lists> ::= <argument-list> <argument-lists> | ""
<arguments> ::= <expression> <arguments> | ""
\end{lstlisting}

\acrshort{bnf} here is extended with the addition of a regular expression (between \texttt{/} delimiters) in the definition of \texttt{<word>}. The regular expression can be read as ``any character which is not white space,
\texttt{[} or \texttt{]}''. This means that aside from white space, which acts as
expression separator there are only two special characters that the parser has to worry about -- the square brackets.

The above grammar definition is obviously very similar to Lisp's BNF description\cite{lisp_bnf_1, lisp_bnf_2}.

The following expression in Lisp:
\begin{lstlisting}
(+ 2 (expt 3 5))
\end{lstlisting}

has an equivalent expression in Dual:
\begin{lstlisting}
+[2 expt[3 5]]
\end{lstlisting}

Comparing these, we may observe that in Dual:
\begin{itemize}
    \item The primary bracketing characters are square brackets (\texttt{[]})
    instead of parentheses. The reason for that design choice is that these
    are easier to type than parentheses or curly brackets (as they do not
    require holding the shift key), which matters considering the ubiquity of
    these characters in the source code.
    \item Expression's operator name is written \textit{before} the opening bracket that precedes the list of arguments, as in \texttt{operator[argument-1 argument-2 ... argument-n]}.
\end{itemize}

Other than these two differences, Dual's notation is equivalent to S-expressions. Its advantages are:
\begin{itemize}
    \item It is easier to parse by a human. Operators are clearer distinguished from operands. This is arguably because this notation is more familiar, bearing a similarity to the general mathematical notation (as in \texttt{f(x)}) and the most popular programming language syntax -- the C-like syntax\footnote{11 out of the
        top 20 languages as of June 2016\cite{tiobe} have C-based syntax (by this classification: \cite{c_family_list_wikipedia}). If
        we extend the syntax family to Algol-like, its virtually 20 out of 20 --
        \cite{pl_genealogy}. There are no languages with Lisp-based syntax among the most popular ones.}
    \item If an expression has another expression as its operator, it is written
      as \texttt{op[args-1][args-2]}, which reduces the amount of nesting and
      thus the amount of identical bracketing characters appearing next to each other in the source code. Compare the equivalent S-expression: \texttt{((op args-1) args-2)}; and with multiple levels:
      \texttt{op[args-1][args-2][args-3][args-4]} vs \texttt{((((op args-1)
        args-2) args-3) args-4)}.
\end{itemize}

An interesting property of this syntax that, depending on the context, could be
classified as an advantage, disadvantage or neither is that the sequence of
characters \texttt{[[} is not legal, whereas in Lisp the analogous sequence \texttt{((} is.

Alas, this simple notation doesn't do away with a lot of other problems inherent
in all minimal syntaxes, related to their homogenity. Later in this chapter I will introduce extensions and syntax sugar, which make the notation a little bit more diverse. Keep in mind that every special character that is introduced, is taken away from the set of possible \texttt{<word>}-characters, which implies that the regular expression for \texttt{<word>} is changed accordingly.

\subsection{Comments}\label{sub:comments}
Comments are a basic and indispensable syntax feature of any programming language. I chose to include a comment syntax similar to the one found in Ada, Haskell or Lua:
\begin{lstlisting}
-- a comment that extends until the end of the line

-- an expression that computes square root of 81:
sqrt[81]

--[
    this is a multiline comment
    --[
        multiline comments can be nested
        
        as long as [ and ] are balanced, anything can be nested within
        multiline comments
        
        for example:
        --[
            this is a comment that includes a piece of code:
            *[7 7]
            
            which would evaluate to 49
        ]
    ]
]
\end{lstlisting}

\subsection{Numbers}
Numbers in the language are represented as JavaScript numbers. This means that
there's only one number type -- 64-bit floating point\footnote{Defined by the
  ISO/IEC/IEEE 60559:2011 (IEEE 754) standard:
  \cite{60559_2011, js_numbers}}. They are implemented as follows:
\begin{itemize}
    \item When a word is tokenized by the parser, it is converted to a
      JavaScript number with a Number type constructor, which returns either the
      corresponding value (if the word is parsable to a number) or the value
      \texttt{NaN}. In the former case, the numerical value is stored in the
      appropriate syntax tree node, as its \texttt{value} property.
    \item Upon evaluation, a syntax tree node is checked for the \texttt{value}
      property. If it has one it is given as the result of the evaluation.
    \item The fact that a number is stored as a syntax tree node, which contains
      the its string representation and its raw value, both obtained from the
      source code during parsing means that conversion from a number literal to
      string is zero-cost, which could be useful for optimization.
\end{itemize}

Thus all of the following JavaScript number literals are valid in Dual:
\begin{lstlisting}
1
357
3.14
0x11 // hexadecimal
0b11 // binary
0o11 // octal
5e-2 // exponential notation
\end{lstlisting}

\section{Escape character}
An escape character \texttt{\\} is introduced. It allows special characters to be included in variable names.

For example:
\begin{lstlisting}
word\ with\ spaces\ and\ braces\[\]
\end{lstlisting}
would be a single valid word and could be used as an identifier for a variable.

\section{Strings}
String values are introduced in Dual as follows:
\begin{lstlisting}
'[A quick brown fox jumps over the lazy dog]
\end{lstlisting}

\texttt{'} is a special operator that produces a string value. It takes any number of arguments, which must be valid expressions.

Strings support variable substitution (also known as string interpolation\cite{string_interpolation_wikipedia}). Assuming we have a variable \texttt{animal-0} with the string value \texttt{"bear"} and another variable \texttt{animal-1} with the string value \texttt{"duck"}, this string:
\begin{lstlisting}
'[A quick brown {animal-0} jumps over the lazy {animal-1}]
\end{lstlisting}

would evaluate to:
\begin{lstlisting}
"A quick brown bear jumps over the lazy duck"
\end{lstlisting}

Special characters inside string can be escaped with the escape character \texttt{\\}. Note that balanced square brackets that are part of syntactically valid Dual expression do not have to be escaped.

The implementation of strings is explained in detail in Section \ref{sec:est}. String interpolation is explained in Section \ref{sec:sub}.

\section{Basic primitives and built-ins}\label{sec:primitives}
This section enumerates and briefly describes Dual's basic primitives and built-in functions and values.

The items in Sections \ref{sub:fun} and \ref{sub:primitives} are structured as follows:
\begin{itemize}
    \item \texttt{<<name>> [<<arguments>>]}
    
    <<description>>
\end{itemize}

Where \texttt{<<name>>} is the name of the function/primitive and
\texttt{<<arguments>>} are either the names that describe the arguments of the
function/primitive or its arity. That is, the number of arguments that the
function/primitive is defined for. This can be a fixed value (e.g
\texttt{1}), a fixed range of values (e.g. \texttt{0..3}) or a range of
values without an upper bound (e.g. \texttt{0..*}, which means 0 or more). An argument name can optionally contain a colon character \texttt{:}, which is followed by the type that the argument is expected to have.

<<description>> is a brief description of the function/primitive.

\subsection{Functions}\label{sub:fun}
The following basic functions are defined in the language:
\begin{itemize}
    \item \texttt{list [0..*]}
    
    Returns a JavaScript \texttt{Array}\cite{mdn_array}, which contains the values of its arguments.
    
    \item \texttt{\$ [0..*]}
    
    An alias for \texttt{list}.
    
    \item \texttt{apply [f args]}
    
    Works like Lisp's \texttt{apply}: it takes a function and a list of arguments and returns the result of applying the function to the arguments.
    
    \item \texttt{log [0..*]}
    
    Wraps JavaScript's \texttt{console.log} method\cite{mdn_log}. It outputs the values of its arguments to JavaScript's standard output -- web browser's console.
    
    \item \texttt{typeof [arg]}
    
    Wraps JavaScript's \texttt{typeof} operator. ``[R]eturns a string indicating the type of the unevaluated operand.''\cite{mdn_typeof}
    
    \item \texttt{or [a b]} and \texttt{and [a b]}
    
    The basic logical operators -- analogous to \texttt{||} and \texttt{\&\&} in JavaScript.
    
    \item \texttt{any [0..*]} and \texttt{all [0..*]}
    
    Like the above, but accept variable number of arguments. These return either \texttt{true} or \texttt{false}.
    
    \item \texttt{not [arg]} 
    
    The logical negation operator (\texttt{!}).
        
    \item \texttt{mod [arg]}
    
    The modulus (\texttt{\%}) operator.
    
    \item \texttt{-# [arg]} 
    
    The unary minus operator. It negates its argument.
    
    \item \texttt{sum [0..*]} and \texttt{mul [0..*]}
    
    Perform summation and product operations on any number of arguments.
    \item \texttt{to-int [arg:number]} 
    
    Converts its argument to an integer value, by truncating the decimal part.
    
    \item \texttt{strlen [str:string]}
    
    Returns the length of \texttt{str}.
    \item \texttt{str@ [str:string n:integer]} 
    
    Returns the \texttt{n}th character of \texttt{str}.
\end{itemize}

Moreover, all basic binary inequality operators \texttt{<, >, <=, >=} as well as all basic binary arithmetic operators \texttt{+, -, *, /} are supported. Comparison operators: \texttt{=} and \texttt{<>} are equivalent to JavaScript's \texttt{===} and \texttt{!==}, which means they perform strict comparison, without implicit type conversion\cite[Section~Equality operators]{mdn_comparison_operators}.

\subsection{Language primitives}\label{sub:primitives}
The Dual language supports the following primitives:
\begin{itemize}
    \item \texttt{bind [name value]}
    
    Evaluates its second argument and binds this value to the name of the first
    argument. This name is bound within the current scope. This is a basic
    construct for defining variables, like \texttt{var} or \texttt{define} in
    other languages. Significant semantics here are that new scopes are
    introduced by function bodies, macro bodies and match expression bodies. The
    primitive also supports pattern matching to deconstruct the value and bind
    its components to possibly several variables. In that regard it works a lot
    like JavaScript's destructuring
    assignment\cite{mdn_destructuring}
    or similar features in other languages, such as Perl or Python. This
    primitive can be used only for binding names that don't exist in the scope
    at the point of its invocation. There are other constructs for mutating and
    modifying existing variables. There is no
    hoisting\cite[Section~var hoisting]{mdn_var},
    as definitions are processed in order in which they appear in code.
    
    For example:
\begin{lstlisting}
bind [greeting '[Hello]]
\end{lstlisting}
    
    \item \texttt{if [condition consequent alternative]}
    
    This primitive serves as a basic conditional evaluation construct. Its
    semantics are like those of the analogous construct in Lisp. It accepts 3
    arguments: first the \texttt{condition} expression, then the
    \texttt{consequent}, that is, the expression to be evaluated if the value of
    the condition is \textit{not false} (note that this is a strict rule; any
    other value than \texttt{false} is interpreted as \texttt{true}; every
    conditional construct in the language follows this rule). The third
    argument, the \texttt{alternative} is the expression that is evaluated
    otherwise.
    \item \texttt{do [0..*]}
        
    Evaluates its arguments in order and returns the value of the last argument.
    
    \item \texttt{while [condition body]}
    
    A basic loop construct. If \texttt{condition} is equivalent to \textit{not
        false}, evaluates \texttt{body}. Repeats these steps until
    \texttt{condition} evaluates to \texttt{false}. Returns the value of the
    last evaluation of \texttt{body} or \texttt{false} if the body was not
    evaluated.
    
    \item \texttt{mutate [name value]}
    
    If a variable identified by \texttt{name} is defined within the current
    scope or any outer scope, changes (mutates) its value, so it now refers to
    the result of evaluating the \texttt{value} argument. The scopes are
    searched from the innermost to the outermost, in order. If the \texttt{name}
    argument doesn't identify any variable, an error is thrown. Returns the
    scope (environment), in which the primitive was evaluated.
    
    \item \texttt{dict [0..*]}
    
    Creates and returns a JavaScript object. It takes an even number of arguments. Arguments are considered in twos, as key-value pairs. These pairs determine properties for the new objects. Keys, which must be words, are property names and their corresponding values, which can be arbitrary expressions, are the values of these properties.
    
    For example:
\begin{lstlisting}
-- creates an object with four properties and assigns it
-- to variable `car`:
bind [car dict [
    id 0
    brand '[Ford]
    model '[Mustang]
    year 1969 
]]
\end{lstlisting}
    
    \item \texttt{assign [2..*]}
    
    A wrapper for JavaScript's Object.assign()\cite{mdn_assign}. It copies the values of all properties from one or more source objects to a target object. Returns the target object. The first argument is the target object, the following arguments are the source objects.
    
    \item \texttt{code' [arg]}
    
    Returns its argument without evaluating it. Used in macros, which return unevaluated code, which is substituted in the syntax tree and only then evaluated.
    
    \item \texttt{macro [1..*]}
    
    Returns a macro value. The last argument is the macro's body. The preceding arguments are the patterns for the macro's arguments. See Sections \ref{sec:pat} and \ref{sec:mac} for details.
    
    \item \texttt{of [1..*]}
    
    Returns a function value. The last argument is the function's body. The preceding arguments are the patterns for the macro's arguments. See Section \ref{sec:pat} for details.
    
    \item \texttt{procedure [body]}
    
    Returns a function value. Its only argument is the function's body.
    
    \item \texttt{match [2..*]}
    
    Performs pattern matching. Its first argument is an expression to be matched. The following arguments are two-element lists, where the first element is the pattern to be matched and the second the expression to be evaluated if it matches. See Section \ref{sec:pat} for details.
        
    \item \texttt{cond [1..*]}
    
    It works like a nested \texttt{if-else}s and similarly to Lisp's \texttt{cond}\cite[Section~5.3, Macro~COND]{common_lisp_hyperspec}. Its arguments are two-element lists, where the first element is a condition that should evaluate to a boolean value and the second is the expression to be evaluated if it the condition is true. It evaluates at most one expression: the one that has a true condition. Conditions are checked in order.
    
    \item \texttt{. [2..*]}
    
    Property accessor. Essentially works like JavaScript's \texttt{.} operator\cite{mdn_dot}:
    
    For example:
    \begin{lstlisting}
    .[window Date now][]
    \end{lstlisting}
    
    translates to JavaScript as:
    \begin{lstlisting}
    window.Date.now();
    \end{lstlisting}
    
    If a property cannot be accessed, an error is thrown.
    
    \item \texttt{: [3..*]}
    
    Works symmetrically to \texttt{.} -- it sets a property to a value specified by its last argument.
    
    For example:
    \begin{lstlisting}
    :[game-state hero ammo 5]
    \end{lstlisting}
    
    translates to JavaScript as:
    \begin{lstlisting}
    gameState.hero.ammo = 5;
    \end{lstlisting}
       
    If a property cannot be accessed, an error is thrown.
    
    \item \texttt{@ [arg]}
    
    Identity operator. Returns the value of its argument.
    
    \item \texttt{async [1..*]}
    
    Its first argument should be an asynchronous JavaScript function, such as \texttt{requestAnimationFrame}\cite{mdn_requestanimationframe}. It applies this function to its remaining arguments.
\end{itemize}

\subsection{Values}
The following values are also defined:
\begin{itemize}
    \item \texttt{true} and \texttt{false}
    
    Evaluate to their respective boolean values. \texttt{\_} is an alias for \texttt{true} when used outside of pattern-matching. This enables a convenient compatibility between \texttt{match} and \texttt{cond}: if we're matching a single value and want to have a default case, then \texttt{\_} is used to match any value. Similarly, if \texttt{\_} is given as a condition in the last alternative of \texttt{cond}, it will evaluate to \texttt{true} and work as the default case.
    
    \item \texttt{undefined}
    
    Evaluates to JavaScript's \texttt{undefined} value. It is a primitive type that is used by the language to mark values that have not been assigned a value. Also, functions that do not explicitly return a value, return \texttt{undefined}\cite{mdn_undefined}.
    
    \item \texttt{window} 
    
    Provides access to JavaScript's global \texttt{window} object\cite{mdn_window}.
\end{itemize}


\section{Enhanced Syntax Tree}\label{sec:est}
In order to enable full mapping between any number of program representations at the syntax-level, a modification of an \acrshort{ast} was designed as a data structure representation of Dual's syntax. I call this structure the \acrlong{est} (\acrshort{est}). This crucial element in the language's design is described in this section.

The primary representation of a program in Dual is the EST. Although itself not directly editable, it can contain references to any number of editable representations, such as the text and visual ones.

These other representations contain back-references to the EST. Thanks to this, a change to any of the representations can be propagated to every other representation.

Every representation must come with:
\begin{itemize}
    \item A way to translate it to an EST.
    \item A way to generate it for a given EST.
    \item A way or ways to manipulate it.
\end{itemize}

While translating, generating and manipulating, it must be ensured that each entity of the representation has a bidirectional association to a corresponding EST node.

For example, for text representation:
\begin{itemize}
    \item Translation to EST is done with a parser.
    \item Generation from EST is done with an unparser\cite{unparser_wikipedia}.
    \item Manipulation is done with a text editor.
\end{itemize}

To ensure that the associations are kept, there must be objects that represent ``text fragments''. These objects then must contain references to corresponding EST nodes and vice versa.

White space characters and comments have no semantic significance, unless serving as separators could be considered one. After parsing, bracketing characters also serve no purpose and can be safely discarded, without influencing the meaning of the program. This is indeed done when constructing an \acrshort{ast} from text in most programming languages.

In case of Dual though, no characters are discarded. Instead, white space, comments, brackets and any other characters are included in the EST, connected to appropriate nodes. Storing all characters in the EST means that the entirety of text representation, in structural form, is accessible straight from the syntax tree. This allows an unparser to recreate it \textit{exactly}.

Such design greatly simplifies the implementation of and integrates with
the language the following features:
\begin{itemize}
    \item Automatic indentation
    \item Documentation comments. Comments can easily be associated with
      corresponding code blocks (syntax tree nodes), which can be useful for
      automatically generating documentation in any format
    \item Any expression can be unparsed to its original form straight from
      syntax tree, which can be used for debugging
    \item This also means that any expression can be stringified on-the-fly and
      this string can be used as a value in the program. This feature allowed me
      to completely omit definition of strings at the parser level, although
      this is not a very efficient solution. Nevertheless keeping strings in
      such structural form -- as syntax trees -- in combination with pattern
      matching enables language-native structural manipulation of
      strings\footnote{See:
        \cite{wolfram_string_patterns}
        and
        \cite[Section~Pattern matching and strings]{pattern_matching_wikipedia}
        for similar concepts.}. For example
      we could write:
    \begin{lstlisting}
        bind [str '[A quick brown fox jumps over the lazy dog]]
                    
        bind [words [_ _ third-word {rest}] str]
        
        bind [characters [_ _ third-letter {rest}] third-word]
        
        -- logs "o" to the console: log [third-letter]
    \end{lstlisting}
    
Where \texttt{words} deconstructs a string into single words and binds these
words to identifiers provided as its arguments and \texttt{characters} performs
an analogous operation on the single character-level. The notation
\texttt{\{rest\}} matches zero or more arguments (see Section \ref{sub:rest} for
details). \texttt{log} outputs the values of its arguments to the JavaScript
console.

 An obvious and very simple optimization
would be to keep the raw form of the string as a value in the
corresponding syntax tree node. Having these two forms alongside each
other would enable the programmer to use the familiar string
manipulation methods as well as structural manipulation.
\end{itemize}

\subsection{Zero and single argument expressions}
In order to reduce the amount of \textit{closing} brackets appearing next to
each other in program's text, two additional simple notations were
introduced. The first is addition of the pipe special character
(\texttt{|}). This character is used for single-argument expressions, as in:
\js{Najpierw należałoby wyjaśnić jak to działa, a dopiero potem dawać przykłady.
  Swoją drogą warto napisac że | wiąże od prawej, tzn. foo | bar | baz to to
  samo co foo [bar [baz]]}

\js{jestem za dopracowaniem wyglądu listingów - czcionka o stałej szerokości
  mile widziana.}
\begin{lstlisting}
    -- compute factorial of 32: factorial|32 -- equivalent to factorial[32]
    
    -- find 9th Fibonacci number: fibonacci|9 -- <=> fibonacci[9]
    
    -- compute sine of pi sin|pi -- <=> sin[pi]
    
    -- compute cosine of the number that is the result of multiplication of pi
    -- and 5!: cos|*[pi factorial|5] -- <=> cos[*[pi factorial[5]]]
    
    -- convert 33.2 to an integer (truncate .2): to-int|33.2 -- <=> to-int[33.2]
    
    -- construct a list with one item, which is a string "hello" list|'|hello --
    -- <=> list['[hello]]
\end{lstlisting}

The above example shows that if a function is invoked with one argument, we can
omit the closing brace and replace the opening brace with \texttt{|}. The parser
produces equivalent syntax tree.

Another special character (\texttt{!}) was introduced for analogous use for
zero-argument expressions (procedures):
\begin{lstlisting}
    -- invoke a procedure that changes some state variables in its outer scope:
    -- set-initial-state! -- <=> set-initial-state[]
    
    -- sum two random numbers: <=> +[random[] random[]]: +[random! random!]
    
    -- bind a value returned by an immediately invoked procedure to an
    -- identifier <=> bind [forty-two procedure [42][]] bind [forty-two
    -- procedure [42]!]  forty-two -- evaluates to 42
\end{lstlisting}


\subsection{Pattern matching}\label{sec:pat}
A simple, yet powerful pattern-matching facility was added to the language.

Pattern matching works with bindings, functions (although the primary
function-producing expression in the prototype doesn't use it by default),
\texttt{match} primitive and macros (not available in the prototype).

The pattern matching works in a way similar to most other languages that support
this feature (e.g. ML family). The general rules are\footnote{For brevity I
  assume here that `of` is a primitive that works like described in
  \ref{subsub:macros}, where the alternative argument is optional. This is how
  it was implemented in an early prototype of the language. If no viable
  alternative was present, an error was thrown.}:
\begin{itemize}
    \item A literal (strings or numbers are supported) value matches itself:
    \begin{lstlisting}
        -- computes factorial of a number bind [factorial of [0 1 of [n *[n
                factorial[-[n 1]]]]] ]
        
        -- logs `120`: log [factorial|5]
    \end{lstlisting}
    \item An identifier (word) matches any value, which is then bound to the
      identifier:
    \begin{lstlisting}
        bind [simple-print of [x log|x]]
        
        -- logs `3`: simple-print[3]
    \end{lstlisting}
    \item A wildcard pattern (\texttt{\_}) matches any value, but doesn't bind:
    \begin{lstlisting}
        -- returns its third argument, discards the rest: bind [get-third of [_
        -- _ x x]]
        
        -- logs `3`: log [get-third[1 2 3]]
    \end{lstlisting}
    As such it can be useful for discarding some values, depending on other
    values or extracting some values from a structure (see next point).
    \item The following expression-patterns are supported:
    \begin{itemize}
        \item \texttt{list} or \texttt{\$} is used to destructure lists:
        \begin{lstlisting}
            bind [$[_ _ third-element] $[0 1 2]]
            
            -- logs `3` log [third-element]
            
            -- it works for arbitrarily nested lists as well bind [ $[ _ $[ _
            -- pick _ _] _] $['|a $['|b '|c '|d '|e] '|f] ]
            
            -- logs `c`: log [pick]
        \end{lstlisting}
        \item Comparison operators (\texttt{= < <= >= <>}) match if a value
          passes the comparison; it can be viewed as a shorthand notation for
          simple
          guards\cite[Chapter~Pattern Matching Basics, Section Using Guards within Patterns]{f_sharp_wikibooks}:
        \begin{lstlisting}
            -- returns the sign of a number note: `-#` is the unary `-`
            -- operator: bind [sign of [=|0 0 of [<|0 -#|1 of [>|0 1]]] ]
            
            -- logs `-1`: log [sign|-77]
        \end{lstlisting}
        \item Other pattern-expressions are not supported and using them will
          result in a mismatch.
    \end{itemize}
\end{itemize}

The above examples show pattern matching used for destructuring values and
binding their components to identifiers and for function definitions. There's
also a \texttt{match} primitive, which can serve the role of a \texttt{switch}
statement from C-like languages. Although pattern matching makes it much more
powerful than that, as any values supported by the pattern matching system can
be matched, including lists, which allow us to switch on multiple values and in
any combinations.

The \texttt{match} primitive's first argument is a value to match and all
subsequent arguments are two-element lists, where the firs element is the
pattern to match and the second is the expression to evaluate in case of a
match. The primitive tries the matches in order and only evaluates the
expression, related to the successful match, which is the first one that
matches. The subsequent matches are not evaluated.

\begin{lstlisting}
    bind [state '|game-on]
    
    -- will execute the `play` procedure: match [state $['|game-on play!]
      $['|game-paused display-pause-menu!]  $['|game-screenshot
        capture-screenshot!]  ]
    
    -- ...

    -- note: . is the access operator .[a b c] is equivalent to a.b.c in other
    -- languages bind [$[x y] .[player postion]]
    
    -- we can easily replace complex conditions: match [$[x x y y] $[ $[>|0
          <|screen-width >|0 <|screen-height] log|'[player visible] ] $[_
        log|'[player not visible]] ]
    
\end{lstlisting}
\begin{itemize}
    \item 
\end{itemize}

With an arsenal of these few simple pattern matching tools we can use a lot of
useful features, which further add expressivity to the language. We can also
imagine many possible extensions and generalizations, as briefly discussed in
Chapter \ref{chap:design}.

\begin{itemize}
    \item Destructuring assignments or, more precisely, destructuring
      definitions.\footnote{Destructuring could easily be extended to mutation
        as well, although I have found it sufficient to be usable only in
        definitions, while implementing the prototype.}. An example of such
      definition would be:
    \begin{lstlisting}
        bind [$[a b $[c d]] $[1 2 $[3 4]] bind [ $[ _ x y { rest }] $['|a '|b
              '|c '|d '|e '|f] ]
        
        -- logs `1 2 3 4`: log [a b c d]
        
        -- logs `b c ["d", "e", "f"]`: log [x y rest]
    \end{lstlisting}
\end{itemize}


\subsection{Rest parameters and spread operator}\label{sub:rest}
\js{Przyznaję, ze czytałem szybko, ale nie zrozumiałem do końca tych operatorów.}
Another syntax extension that I introduced involved two additional special
bracketing characters: \texttt{\{} and \texttt{\}}, which serve several
purposes:
\begin{itemize}
    \item Rest parameters mechanism known from
      Lisp\cite[Section~12.2.3]{emacs_lisp_reference},
      recently also adopted in JavaScript (as of the ECMAScript2015
      standard\cite{mdn_rest}). That
      is, for example:
    \begin{lstlisting}
        bind [variadic-function of [a b {args} log [a b args] ]]
        
        -- logs `1 2 [3, 4, 5, 6]`: variadic-function[1 2 3 4 5 6]
    \end{lstlisting}
    This enables the user to easily define variadic functions, which can be
    called with a variable number of arguments.  This works in any place, where
    pattern matching works:
    \begin{lstlisting}
        bind [$[a b {rest}] $['|a '|b '|c '|d '|e]]
        
        -- logs `["c", "d", "e"]` log [rest]
    \end{lstlisting}
    
    thus enabling non-exact matching.
    
    \item Spread operator (also inspired by the analogous feature from
      ECMAScript2015):
    \begin{lstlisting}
        bind [f of [a b c d e f log [a b c d e f]]] bind [args $[8 7 6]]
        
        -- logs `9 8 7 6 5 4`: f[9 {args} {$[5 4]}]
    \end{lstlisting}
    This provides a much nicer and more powerful alternative to \texttt{apply},
    Lisp's fundamental function, which applies a function to a list of
    arguments. It's a way to flatten any list onto a list of arguments. This
    works for multiple values and lists as well:
    \begin{lstlisting}
        -- alternative way to achieve the same result as in the previous listing
        -- logs `9 8 7 6 5 4`: f[{9 args $[5 4]}]
    \end{lstlisting}
    \item String interpolation notation:
    \begin{lstlisting}
        bind [name '|Bill]
        
        -- logs `Hello, Bill.` log ['[Hello, {name}.]]
    \end{lstlisting}
    
    As we can see this gives us a very convenient notation for string
    interpolation, similar to e.g. template literals in
    JavaScript\cite{mdn_template_strings}.
    In order to escape curly braces, they should be doubled:
    \begin{lstlisting}
        -- logs `Hello, {name}.` log ['[Hello, {{name}}.]]
    \end{lstlisting}
    
    I also added a special type of string -- an HTML string, where interpolation
    notation is the other way around -- double braces cause substitution, single
    braces do nothing:
    \begin{lstlisting}
        bind [name '|Bill] -- logs `<h1>Hello, Bill.</h1>` log [html'[<h1>Hello,
            {{name}}.</h1>]]
        
        -- logs `<h1>Hello, {name}.</h1>` log [html'[<h1>Hello, {name}.</h1>]]
    \end{lstlisting}
    
    This is to enable embedding CSS and JavaScript code inside those strings,
    without having to constantly escape brace characters.
    
    \item Unquote notation for macros\footnote{Analogous to \texttt{unquote} or
      \texttt{,} in Lisp:
      \cite[Section~1.3.8]{racket_reference}
      -- see Chapter \ref{chap:design}}.
\end{itemize}

\section{Just-in-time macros}\label{sec:mac}
One feature that I experimented with while creating the prototype of Dual is
support for ``first-class just-in-time expanded macros''. By this I mean macros
similar to those found in Lisp, but with a few key characteristics, which
differentiate them from conventional implementations of macros in Lisp-like
languages. These are:
\begin{itemize}
    \item Macros are expanded upon evaluation; when a macro invocation is
    encountered by the interpreter, it is expanded into code; the node in
    the EST containing the macro invocation is permanently replaced by the
    expanded expression, which is subsequently evaluated and its value is
    returned as the value of the invocation.
    \item Macros can return other macros in a straightforward manner; this
    feature nicely composes with the variation of Lisp syntax found in
    Dual; I used it extensively to improve it, mostly with the goal of
    reducing the amount of adjacent closing brackets in the source code.
\end{itemize}

In order to support first-class runtime macros a Lisp interpreter can be
modified as follows\cite{macros}:
\begin{itemize}
    \item Primitives are moved into the top-level environment; they thus are
    no longer treated as special case by the \texttt{eval} function.
    \item A new primitive, \texttt{macro} is added, which is essentially
    equivalent to \texttt{lambda}, except that it produces macro values
    instead of function values.
    \item The \texttt{apply} function is now responsible for checking the
    type of an expression's operator, which can be a \texttt{primitive}, a
    \texttt{macro} or a normal expression; this determines whether the
    arguments are evaluated before application
\end{itemize}

This results in a simpler, more uniform and at the same time more powerful
interpreter. A major advantage is that:
\begin{quote}
    Because of their first-class nature, first-class macros make it easy to add or
    simulate any degree of laziness\cite{macros}
\end{quote}

The below listing presents an example of a macro named \texttt{if*} that wraps
the \texttt{if} primitive in a slightly different syntax. This syntax wraps the
condition, consequent and alternative parts of the \texttt{if} in separate
blocks delimited by \texttt{[]}. The condition is required to be an infix
expression in the form \texttt{a operator b}. The consequent and alternative
blocks take care of wrapping all expressions within them in \texttt{do}
blocks. This makes it more convenient and less error-prone to write complex
\texttt{if} expressions:
\begin{lstlisting}
bind [if* macro [a op b macro [{then} macro [{else} code'[if [apply[{op} {a}
{b}] do[{then}] do[{else}]]] ]]]

if* [a < b][ log ['[a is less than b]] a ][ log ['[b is less than or equal to
a]] b ]

-- expands to: if [<[a b] do [ log ['[a is less than b]] a ] do [ log ['[b is
less than or equal to a]] b ]]
\end{lstlisting}

%TODO: elaborate


\subsection{In combination with zero and single argument expressions}\label{subsub:macros}
The combination of the macro system and the syntax sugar for zero and single argument expression (\texttt{|} and \texttt{!}) helps reduce the amount of bracketing characters even further.

For example, if we define a \texttt{match*} and \texttt{of*} macros as
described, the following expression:
\begin{lstlisting}
bind [x 99]

-- will log "x is greater than one": match* [x] | of* [<|0] [log|'[x is
negative]] | of* [0] [log|'[x is zero]] | of* [1] [log|'[x is one]] |
log|'[x is greater than one]
\end{lstlisting}

which is somewhat similar syntactically to
ML-style\cite[Section~Algebraic datatypes and pattern matching]{standard_ml_wikipedia}
languages, could be translated into the following:
\begin{lstlisting}
bind [x 99]

-- will log "x is greater than one": apply [ of [<|0 log|'[x is negative] of
[0 log|'[x is zero] of [1 log|'[x is one] log|'[x is greater than one] ]
] ] x ]
\end{lstlisting}

where \texttt{apply} would be defined analogously to Lisp's
\texttt{apply}. \texttt{of} would be defined as a function primitive with arity
2..*, which treats its penultimate argument as the function's body and all the
preceding arguments as patterns for the function's arguments. The last argument
is used when such a function is called and the values supplied as arguments
don't match the patterns. If the argument is a function, it will be called with
the same values as arguments and if it's a value it will be returned.

I used such a solution for pattern matching in the early prototypes, but
replaced it with a native \texttt{match} construct (described in Section
\ref{sec:primitives}) for performance reasons. Nevertheless this shows that a
few simple, but general syntax rules and a powerful macro system, can be a very
flexible tool for extending syntax.
