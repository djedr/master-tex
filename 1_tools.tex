\chapter{Tools and methods}\label{chap:tools}
The Dual programming language is build on top of the web platform\footnote{\url{https://platform.html5.org/}}. This means that it utilizes web technologies. Specifically:
\begin{itemize}
    \item JavaScript
    \item HTML5. In the scope of this thesis I will touch in detail on the following elements of this standard:
    \begin{itemize}
        \item Canvas element
    \end{itemize}
    \item CSS
\end{itemize}

Other related technologies that are relevant to development of the language are:
\begin{itemize}
    \item JavaScript components that provide code editor features\footnote{\url{https://en.wikipedia.org/wiki/Comparison_of_JavaScript-based_source_code_editors}}, specifically CodeMirror\footnote{\url{http://codemirror.net/}}, which is used as the text editor component component of the Dual programming language editor
    \item Standalone modern code editors, built with web technologies and with web development in mind, such as:
    \begin{itemize}
        \item Visual Studio Code\footnote{\url{https://code.visualstudio.com/}}
        \item Brackets\footnote{\url{http://brackets.io/}}
        \item Atom\footnote{\url{https://atom.io/}}
    \end{itemize}
    \item Node.js\footnote{\url{https://nodejs.org/en/}}, a JavaScript-based runtime environment for developing server-side web applications
\end{itemize}

Also an important part of this research are visual programming languages in general\footnote{\url{
http://blog.interfacevision.com/design/design-visual-progarmming-languages-snapshots/}} and [[]] their development environments.
In particular Blueprints Visual Scripting system in Unreal Engine 4\footnote{\url{https://docs.unrealengine.com/latest/INT/Engine/Blueprints/}}.

In developing the Dual programming language 
short
Web platform
JavaScript
HTML
CSS

Browser environment

Node.js

\section{Programming discipline}
Primarily exploratory programming:
``exploratory programming: the kind where you decide what to write by writing it.''\footnote{\url{http://arclanguage.org/}}