\chapter{Tools and methods}\label{chap:tools}
% THIS CHAPTER SHOULD PROBABLY BE MERGED WITH THE NEXT ONE

This chapter briefly introduces the theoretical and practical components involved in design and implementation of the Dual programming language and its environment. I may further use the terms ``Dual system'' or simply ``system'' to refer to these.

\section{Web technologies}
One of the main goals in designing the system is accessibility. This is accomplished in practice by building it on top of arguably the most accessible and ubiquitous platform -- the web platform\footnote{\url{https://platform.html5.org/}}.

The language's interpreter and development environment is intended to work with and is built on web technologies. Specifically JavaScript, HTML5 and CSS. The prototype implementation makes use of Node.js -- server-side JavaScript runtime and CodeMirror\footnote{\url{http://codemirror.net/}} -- a JavaScript library which provides basic facilities for the text-based code editor part of the system. This part is modeled after modern web-oriented code editors with similar design philosophy\footnote{\url{https://en.wikipedia.org/wiki/Comparison_of_JavaScript-based_source_code_editors}}, such as Visual Studio Code\footnote{\url{https://code.visualstudio.com/}}, Brackets\footnote{\url{http://brackets.io/}}, Atom\footnote{\url{https://atom.io/}} and many others.

The design of Dual's visual representation draws from many visual programming languages\footnote{\url{
http://blog.interfacevision.com/design/design-visual-progarmming-languages-snapshots/}}. Analyzing these, we can observe many distinct approaches of which two particular designs are the most widespread and successful. These can be described as line-connected block-based and snap-together block-based visual languages. The former family is exemplified by the Blueprints Visual Scripting system of Unreal Engine 4\footnote{\url{https://docs.unrealengine.com/latest/INT/Engine/Blueprints/}} and the latter by MIT Scratch\footnote{\url{https://scratch.mit.edu/}}.

\section{Programming languages}
Lisp family
Pattern matching languages (ML-family?)
JavaScript





\section{Programming discipline}
The prototype was implemented largely in the spirit of exploratory programming: ``the kind where you decide what to write by writing it.''\footnote{\url{http://arclanguage.org/}}.

This approach in combination with a dynamic and flexible language like JavaScript enables one to quickly transform ideas to working prototypes and shape them as one goes along. But the usefulness of this method is limited, as it may quickly produce fairly low-quality code, as it is not focused on future maintainability.

\comment{
An example application built with the language 

Canvas element

Other related technologies that are relevant to development of the language are:
\begin{itemize}
    \item JavaScript components that provide code editor features, specifically CodeMirror, which is used as the text editor component component of the Dual programming language editor
    \item Standalone modern code editors, built with web technologies and with web development in mind, such as:
    \begin{itemize}
        \item Visual Studio Code\footnote{\url{https://code.visualstudio.com/}}
        \item Brackets\footnote{\url{http://brackets.io/}}
        \item Atom\footnote{\url{https://atom.io/}}
    \end{itemize}
    \item Node.js\footnote{\url{https://nodejs.org/en/}}, a JavaScript-based runtime environment for developing server-side web applications
\end{itemize}
In developing the Dual programming language 
short
Web platform
JavaScript
HTML
CSS

Browser environment

Node.js
}