\chapter{Prototype implementation}\label{chap:impl}

\section{Programming discipline}
The prototype was implemented largely in the spirit of exploratory programming:
``the kind where you decide what to write by writing
it.''\cite{arc}.

This approach in combination with a dynamic and flexible language like
JavaScript enables one to quickly transform ideas to working prototypes and
shape them as one goes along. But the usefulness of this method is limited, as
it may quickly produce fairly low-quality code, as it is not focused on future
maintainability.

Most of the features of the language and editor in the prototype are implemented
as a proof-of-concept, although some are more refined than others in order to
fulfil the major goals of this thesis, one of which was to implement a working
non-trivial application in the language.

\section{The language}
The prototype implementation of the language contains all the features described in Chapter \ref{chap:lang}, with the following exceptions:
\begin{itemize}
    \item Macros are not implemented.
    \item There are two primitives, which produce function values: \texttt{of} and \texttt{of-p}. The second has the same meaning as \texttt{of} described in Chapter \ref{chap:lang}. The first has the same meaning, except that it does not use pattern matching when binding names to arguments. This primitive requires that all the names must be words.
    \item Comments are treated as a streams of characters, taking into account nesting and balancing of brackets in multi-line comments, but are not preserved on the \acrshort{est} as a tree-like structure.
    \item Strings only have a structural representation. They are represented as syntax tree nodes. The performance penalty is acceptable in the prototype implementation.
    \item The escape character \texttt{\\} has no special meaning. To substitute a special character in a string, the following built-in values are defined:
    \begin{lstlisting}
    (left-bracket) -- escapes "["
    (right-bracket) -- escapes "]"
    (left-brace) -- escapes "{"
    (right-brace) -- escapes "}"
    (pipe) -- escapes "|"
    (bang) -- escapes "!"
    \end{lstlisting}
    
    So \texttt{'[(left-bracket)hello(right-bracket)]} would evaluate to: \texttt{"[hello]"}.
\end{itemize}

Lisp's syntax is as minimal as it gets\cite{syntaxation}, which makes 

An interpreter for Lisp is also trivial to implement, so this is a good starting
point.

There are many approaches to implementing interpreters for LISP in
JavaScript\cite{js_lisps}, but the general principles are the same.

\subsection{Macros}
This
macro system is not included in the final version of the prototype, although a
proof-of-concept of it that I implemented in earlier prototypes

\section{The environment}

