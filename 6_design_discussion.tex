\chapter{Design discussion}\label{chap:design}

more proposed features

own front-end framework
angular
react
data binding

modern ide

vs code

few general and powerful features
rather than a lot of specific features

the less primitives/axioms the better
Arc
goto

lazy evaluation
    if
    
    
try-bind: explicit pattern matching

editor manipulation

internals of the language available
    first-class environments
    cite wouter
    
\section{Comments}\label{sec:comments}
If multiline comments were implemented as expressions on parser-level then, in combination with \texttt{|} special character we could have one-word comments, which could be useful for describing arguments to facilitate reading of expressions. For example we could implement list comprehensions, where:
\begin{lstlisting}
    $<-[^[x 2] x range[0 10]]
    $<-[$[x y] x $[1 2 3] y $[3 1 4] <>[x y]]
\end{lstlisting}
would be equivalent to Python's\footnote{\url{https://docs.python.org/3/tutorial/datastructures.html#list-comprehensions}}:
\begin{lstlisting}
    [x**2 for x in range(10)]
    [(x, y) for x in [1,2,3] for y in [3,1,4] if x != y]
\end{lstlisting}
As we see this notation is acceptable (if not cleaner) for simple comprehensions, but starts being less readable for complex ones. This could be alleviated by introducing one-word comments:
\begin{lstlisting}
    $<-[^[x 2] --|for x --|in range[0 10]]
    
    $<-[$[x y] --|for x --|in $[1 2 3]
               --|for y --|in $[3 1 4]
               --|if <>[x y]]
\end{lstlisting}
which are easily inserted inline with code and have a benefit of clearly separating individual parts of an expression, because of being easily distinguished visually from the rest. This can simulate different syntactical constructs from other programming languages, like:
\begin{lstlisting}
    if [>[a b] --|then
        log['|greater]
    --|else log['|lesser-or-equal]
    ]
\end{lstlisting}
Except that it is not validated by the parser. But we could imagine a separate or extend the existing syntax analyzer, so it could validate such ``keyword'' comments or even use them in some way. For example, we could add a static type checker to the language -- in a similar manner that TypeScript or Flow\footnote{\url{https://flowtype.org/}} extends JavaScript. This would be completely transparent to the rest of the language, so any program that uses this feature would be valid without it and it could be turned on and off as needed.

To reduce the number of characters that have to be typed, we could decide to use a different comment ``operator'', such as \texttt{\%}:
\begin{lstlisting}
    $<-[^[x 2] %|for x %|in range[0 10]]
    
    $<-[$[x y] %|for x %|in $[1 2 3]
               %|for y %|in $[3 1 4]
               %|if <>[x y]]

    if [>[a b] %|then
        log['|greater]
    %|else log['|lesser-or-equal]
    ]
\end{lstlisting}

Or even, at the cost of complicating the parser, introduce a separate syntax for one-word comments:
\begin{lstlisting}
    -- `%:type` could be a type annotation 
    bind [a 3 %:integer]
    bind [b 5 %:integer]

    -- will print "lesser-or-equal"
    if [>[a b] %then
        log['|greater]
    %else log['|lesser-or-equal]
    ]
\end{lstlisting}

% only single-word comments could matter: http://c2.com/cgi/wiki?HotComments
% or perhaps this could be a different meta-construct, extension mechanism

% arguments: positional, keywords/keys
% lisp keywords

In future versions of the language, comments will be stored separately from whitespace in the EST. This enables easy smart indentation -- only a prefix of the relevant expression has to be looked at, no need to filter out comments. It also enables using comments structurally, as a metalanguage for annotations, documentation, etc.

\section{Module system}


\section{Extensions to parameter pattern matching}

    proposed syntax
        optional parameters {a b c}
        default args {?[a 3] ?[b 4] ?[c 5]}

    elaborate on the duality/symmetry of binding/definition and evaluation
    
    destructuring code for macros
    
    matching arbitrary expressions/types

\section{Structural string manipulation}
\begin{lstlisting}
    words[_ _ _ fourth _ sixth] -- supa fast, out of the box
    characters[_ _ _ _ fifth]
\end{lstlisting}

 \section{First-class macros}
 implemented, but turned off/discarded in the final prototype
 need a rewrite (not within time scope of this thesis)
 
 jit-expanded
 
 work on zero-cost
    for simple cases works, but
    if nontrivial, in loops, nested, etc. then it's a complex problem
    
possibilities:
    compile-time macros
    
even more flexible system:
    read (parse) time macros // link to json in lisp
 
gensym, hygiene
 
 \section{A better evaluation model}
 better suited to the browser environment
 Explicit-stack execution model
 
\section{C-like syntax}
Throughout this thesis I introduced multiple ways in which the basic, Lisp-like syntax of Dual can be easily extended with simple enhancements, such as adding more general-purpose special characters, macros, single-word comments (as described in Section \ref{sec:comments}), etc.

Going further along this path, keeping in mind that a real-world language should appeal to its users 
% syntaxation, emotions
we find ourselves introducing more and more elements of C-like syntax. This section describes more possible ways in which the simple syntax could be morphed to resemble the most popular languages.
% cite the same thing? ref to previous~ chapter?


direct mapping to the bare syntax

make syntax more heterogenous

.[a b c]
a.b.c -- THIS

minimalist approach to language design
Paul Graham's Arc

renamed define to bind
shorter, slightly specific meaning

mental gymnastics and thesaurus at hand

I prefer short but complete words

disadvantage is that changing the name of something fundamental makes it seem unfamiliar
which is not necessary if the only thing that actually changed is the name/syntax and not the semantics

infix notation inside ()
\url{https://en.wikipedia.org/wiki/Shunting-yard_algorithm}


% having added so much complexity already might as well
Ultimately all this could be implemented with a conventional complex parser for a C-like language that translates to bare Dual syntax. []

Below I present a snapshot from one of designs I have been working on in order to achieve some goals described in this section:
\begin{lstlisting}
    fit map" {f; lst} {
    	let {i; ret} [0, []];
    	
    	while ((i < lst.length)) {
    		ret.push f(lst i);
    		set i" ((i + 1))
    	};
    	ret
    };
\end{lstlisting}

This would be equivalent to:
\begin{lstlisting}
    bind ['|map of ['|f '|lst do [
    	bind ['[i ret] $[0 $[]]]
    	
    	while [<[i lst|length] do [
    		ret[push][f[lst|@[i]]] 
    		mutate* ['|i +[i 1]]
    	]]
    	ret
    ]]]
\end{lstlisting}

Using the notation presented in Chapter \ref{chap:lang}.

One may observe that:
\begin{itemize}
    \item The syntax is much richer, somewhat C-like, but with critical differences, reflecting significantly different nature of the language. At a first glance, it has a familiar look defined by blocks of code delimited by curly-braces, inside which statements (actually expressions) are separated by semicolons; there are different kinds of bracketing characters (\texttt{\{\}()[]}) with different meanings (described below)
    \item Names of the primitives are \textit{full} English words, although as short as possible. \texttt{let} introduces a variable definition -- similarly to \texttt{bind}. \texttt{fit <name> <args> <body>} is a shorthand for \texttt{let <name> (of <args> <body>)}, where \texttt{of} produces a function value. This translates to \texttt{bind [<name> of [<args> <body>]]}.
    \item \texttt{\{\}} delimit a string; inside a string words are separated by \texttt{;}. Strings are stored in raw as well as structural (syntax tree) form. They are a way of quoting code. This provides an explicit laziness mechanism. One-word strings are denoted with \texttt{"} at the end of the word, which resembles the mathematical double prime notation.
    \item \texttt{[]} delimit list literals; inside list literals, elements are separated by \texttt{,}. Lists are a basic data structure. They are actually objects, somewhat like in JavaScript. If a list contains at least one \texttt{:} character (not shown in the example), it will be validated as key-value container; if it doesn't, it will be treated as array with integer-based indices
    \item \texttt{()} are used in function invocations; \texttt{f(a, b, c)} translates to \texttt{f[a b c]}; \texttt{,} separates function arguments; \texttt{f x} is a shorthand notation for \texttt{f(x)}. This, in combination with currying primitives into appropriate macros allows for elimination of excessive brackets and separators. Invocations of primitives resemble use of keywords from other lanugages. 
    \item But at the same time primitives are defined as regular functions -- they are no longer treated exceptionally by the interpreter. When they are invoked, all of their arguments are first evaluated. This works, because now it is required that the programmer quote any words that shouldn't be evaluated, such as identifier names when using \texttt{let}. So primitives are just regular functions operating on code, thanks to the explicit laziness provided by strings.
    \item \texttt{(())} introduce an infix expression, which respects basic operator precedence: (\texttt{((a + b * 2))} would translate to \texttt{+[a *[b 2]]}. This could be implemented with a separate parser based on the shunting-yard\footnote{\url{www.cs.utexas.edu/~EWD/MCReps/MR35.PDF}} or similar algorithm that is triggered by the \texttt{((} sequence. It would translate these infix expressions to prefix form and return them back to the original parser.
\end{itemize}

\section{Modules}
In principle
import and module keywor


\section{Editor}
Synchronized scrolling to keep track of the point in code in both representations.

Vault for code fragments
Snippets = expanded macros

Other features
Support for opening files outside of projects.

Experimenting with application architecture. JSON over WebSockets, only generated HTML, Shared Web Workers and other web technologies.


\subsection{Debugging}

There's limited support for debugging in the current version:
debug expression

one of the former iterations had a mechanism for setting breakpoints on individual expressions

%%%%

dynamic scope
http://letoverlambda.com/index.cl/guest/chap3.html
see: Duality of syntax

\section{Future work}
Compilation
    bytecode?
    JS
    WebAssembly

Brendan Eich:
Tool time
    Static type checking
Compile time
Runtime

\section{General considerations}

Pros of minimal syntax

Cons of minimal syntax

Operator precedence
http://www.ozonehouse.com/mark/blog/code/PeriodicTable.pdf
% periodic table of operators in Perl 6