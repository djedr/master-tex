\chapter{Summary and conclusions}\label{chap:summary}
The language and its development environment presented in this thesis is by no means a complete design. It should be viewed as a snapshot from a design process that I intend to continue in the future.

However the effort invested in the project was sufficient to achieve the general goals, listed in Chapter \ref{chap:intro}. This is reflected in:
\begin{itemize}
    \item The core language, which is Turing-complete and thus capable of implementing any algorithm\footnote{In the same sense as JavaScript or C. No existing language is Turing-complete in the absolute sense, because of physical hardware limitations.}. A simple Brainfuck interpreter is included as an example program (see Appendix \ref{app:plyta}) to demonstrate this\cite{bf_turing_complete}. Moreover, the language is based on Lisp, inheriting a lot of its expressive power and extensibility.
    \item The above is also demonstrated in implementing a non-trivial application, which is the clone of Pac-Man described in Chapter \ref{chap:case}. This exercise also showed significant flaws in the prototype related to performance and thus helped set directions for future improvement.
    \item Design and implementation of a mechanism, which enables the direct correspondence and interchangeability of the visual and text representations. In principle any number of representations could be associated.
    \item Implementation of the prototype of the language's development
    environment with integrated text and visual editors. This demonstrates practically the main ideas outlined in Chapter \ref{chap:editor}.
\end{itemize}

Visual programming systems that provide the ability to work with text and visual representations were developed in the past\cite{snapshots}. But none of them seems to have a similar degree of integration as the system presented here.

\clearpage
I am convinced that some of the ideas presented here are a valuable contribution in the area of (visual) programming language design and development and can be of use for current and and future designers in improving the existing as well as creating new, better\footnote{
While doing this research for this thesis I found a VPL project named Luna, that may be built around a similar idea of highly integrated text and visual representations. The project's website claims (emphasis mine):
\cite{luna_website}:
\begin{quote}
Luna is the \textit{world’s first} programming language featuring two exchangeable representations: textual and visual, which you can freely switch at any time.
\end{quote}

However the project is in ``private alpha'' stage -- it is not publicly available as of July 2016. Nonetheless this may be taken as an indication that the ideas presented in this research are a step on a path to better future VPL systems.
} visual language systems.

I learned that programming language design is a tremendous and heroic task, especially if the language being designed is intended to be of any real-world use. Designing and implementing such a language absolutely from scratch, while introducing useful innovation cannot be done within the time limits of research for a thesis, unless perhaps by an experienced language designer. But such experience has to be gained somehow this was an excellent opportunity.
