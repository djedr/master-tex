\chapter{Introduction}\label{chap:intro}

\section{Title}
The term ``Pac-Man-complete'' in the title refers to a somewhat humorous description of the Idris programming language\footnote{\url{http://www.idris-lang.org/}} attributed to the language's author, Edwin Brady\footnote{\url{https://twitter.com/folone/status/494017847585415168}}. In the context of this thesis it means that the designed and implemented programming language is complete enough and provides enough features to allow one to write a clone of the classic Pac-Man game in it.

\section{Scope}
% more on innovation
This research explores various topics in the field of programming language design. Particularly, the possibilities of integrating and combining multiple representations of the same programming language in a dynamic way. I try to find ways of combining features of visual languages with text-based languages by creating a development environment which lets the programmer work with both representations in parallel or intertwine them in any way.

%This ``dual'' representation of the language 
The created language is used to implement a Pac-Man clone. This provides a demonstration of Dual's capabilities and is a reference for assessing the performance of the implementation.

I discuss further possible improvements in the language's design and performance, as well as set down directions for any future research, which I intend to take on. The exploratory nature of this work lets me cover a fairly broad area, connecting programming language design, computer game development as well as web technologies and web application development.

%%
%exploring various features across different languages
% and [[analyze, compare, other things]]
%Approach presented here can be generalized to any number of representations
%The scope of this research
%Exploratory
%I intend to develop this further
%I also look at programming language design in general and try to find establish a balance of features for a modern language

\section{Choice of subject}
The choice of this particular subject stems from my deep personal interest in programming language design. This research is an opportunity for me to create a project that demonstrates various ideas in this area that I developed over time and to refine them further.

%mainly  have been growing in popularity [[citation]]. I intend to further this growth with this research. 

\section{Related work}
%Literature and existing solutions
Visual languages are not especially popular compared to text-based languages. But recently they have been gaining more popularity, particularly in game development. Chief example and the main cause of this is Unreal Engine, the highly popular and mainstream\footnote{\url{https://en.wikipedia.org/wiki/List_of_Unreal_Engine_games}, \url{http://www.guinnessworldrecords.com/world-records/most-successful-game-engine}} game engine, which in the latest version introduced a visual programming language\footnote{\url{https://docs.unrealengine.com/latest/INT/Engine/Blueprints/}} as its primary scripting language. In fact this is the only scripting language that the engine supports, having dropped the UnrealScript language\footnote{\url{http://www.gamasutra.com/view/news/213647/Epics_Tim_Sweeney_lays_out_the_case_for_Unreal_Engine_4.php}} included in the previous versions. 
%Explore the [[unexplored]]

I intend to further the growth of visual programming languages, propose an even more accessible solution and explore possible improvements over comparable technologies.

I create a unique design, which combines [[]] and seems to be, certainly not to this extend and not with these technologies, a previously unexplored ground.

It's also a good text-based language designed with rapid prototyping of computer games and web applications in mind, but fairly applicable as a more general-purpose solution.

Not tied to any commercial product, dependent only on the ubiquitous web platform.

Furthermore ideas regarding programming language design presented here are applicable in general, regardless of technology.

\section{Goals}
In line with the above, the purpose of this work is to introduce innovation as well as show a practical application of the developed solution. The concrete goals are:
%The goal is not to simply go through the motions of implementing a programming language, while changing some superficial aspect of it -- such as syntax -- though this is incidentally done as well. But the point here is exploring some of the fundamental concepts and in language design and trying to innovate there.
\begin{itemize}
	\item Design and implement a programming language, whose \textit{complete} textual representation must be directly and dynamically mappable to a visual representation and vice versa
	\item Explore and establish directions where innovation is possible in programming language design and implementation
	\item Create a prototype of the development environment for the language on top of the web platform
	\item Evaluate the performance of the language by implementing a clone of Pac-Man and comparing it to other implementations
	\item Explore and discuss further refinements and possible future design directions
	\item Compare the language to existing solutions %?
\end{itemize}

\section{Structure}
This thesis is structured as follows:

Chapter \ref{chap:intro} is this introduction.

Chapter \ref{chap:tools} briefly describes technologies and tools used in developing any software described here.

Chapter \ref{chap:background} discusses the essential elements of the theoretical framework upon which the language was built.

Chapter \ref{chap:lang} describes the Dual language: its syntax, semantics and basic design.

Chapter \ref{chap:editor} talks about the architecture, design and serves as a practical documentation of the language editor and visual representation.

Chapter \ref{chap:design} elaborates on language design both in context of Dual and in general.

Chapter \ref{chap:case} describes a non-trivial application developed with Dual: a Pac-Man clone. Performance of the language is assessed [[compared]] and possible adjustments and improvements are discussed.

Chapter \ref{chap:comparison} relates the language with other similar solution, [[Unreal Engine's Blueprint]].

Chapter \ref{chap:summary} summarizes and concludes.


 Czy rozwiązania
istniejące w~danej dziedzinie nie są wystarczające? Czy problem można rozwiązać
inaczej? Czy podejmowany problem jest aktywnym tematem badawczym? Przed jakimi
wyzwaniami stoi osoba podejmująca tematykę? Na tym etapie należy zarysować
problem w~sposób ogólny.

\comment{Cele muszą być sformułowane w~sposób zwięzły i~\textbf{ścisły}.}

\comment{Alternatywnie, zamiast zakładać tutaj cele do realizacji, można
  opisywać wkład pracy dyplomowej w stan wiedzy w danej dziedzinie.  W ten
  sposób czytelnik już na wstępie wie, jakie są osiągnięcia autora.}


W tym podrozdziale należy szczegółowo uzasadnić dlaczego wybrany został taki
a~nie inny temat pracy. Trzeba przede wszystkim zaprezentować aktualny stan
wiedzy w~danej dziedzinie. Oznacza to konieczność omówienia książek
(ew. artykułów naukowych bądź dokumentacji technicznej) z~których będzie się
korzystać w~trakcie rozprawy. Następnie należy wskazać -- tym razem już
konkretnie -- co nowego zamierza się zrobić. Podstawowymi celami tego
podrozdziału jest wprowadzenie czytelnika w~aktualny stand danej dziedziny
i~przekonanie go że \textbf{naprawdę warto zajmować się podjętym tematem}.
