\chapter{Dual programming language design}\label{chap:lang}

\section{Introduction}
To increase the chance of drawing the attention of an average programmer, who composes programs with text, the designed language must have a solid text representation. This way the programmer always has a fallback and can use whichever is convenient.


This chapter describes the text representation of Dual, which is the basis for the executable representation for the interpreter (the syntax tree) and for
any other editable representation -- including the block-based visual
representation implemented in the prototype.



\js{Dalsze akapity mogą zostać.}

The evolution of programming languages is a gradual process. And so is the
process of designing a single language. The approach that I found effective was
iterative refinement, addition, testing, and sometimes subtraction of
features. In practice this translates to intermediate designs and
implementations being rearranged into new forms, with some discarded. I did not
arrive at something that I could call the final form of the language, so a lot
of the features described here are subject to change and improvement. This might
show in descriptions, where along with talking about the prototype
implementation I propose alternatives and refinements.

In chapter \ref{chap:design} I discuss features and ideas that reached only the
design stage and were not implemented, although some of them will be further
researched outside of scope of this thesis.

Most of the features of the language and editor in the prototype are implemented
as a proof-of-concept, although some are more refined than others in order to
fulfil the major goals of this thesis, one of which was to implement a working
non-trivial application in the language. I cover a lot of design and
implementation surface, only delving deep into some features that are relevant
to core ideas that I wanted to convey in this thesis. The other features are
implemented necessarily, as steps along the path to practical application of
those ideas.

For these reasons the created environment is by no means complete and ready for
use in developing complex applications. The degree of completeness of the
project is reflected in:
\begin{itemize}
    \item The core language, which is sufficiently expressive to implement any
      algorithm, i.e. Turing-complete\footnote{In the same sense as JavaScript or C. No existing language is Turing-complete in the absolute sense, because of physical hardware limitations.}. A simple Brainfuck interpreter is included as an example program (see Appendix \ref{app:plyta}) to demonstrate this\cite{bf_turing_complete}.   
    \item Implementation of several interesting additional features of the core
      language, which are described in this chapter \js{wymienić jakich}
    \item The ablility of the language to implement also non-trivial
      applications. This is demonstrated by implementing a clone of Pac-Man,
      described in Chapter \ref{chap:case}
    \item The direct correspondence of the visual and text representations, with
      the possibility of parallel dynamic editing; albeit the visual editing
      part of the editor includes only basic features and is not optimized in
      terms of performance; details are described in Chapter \ref{chap:editor}
    \item Implementation of the prototype of the language's development
      environment on top of the web platform, which includes a server-side and a
      client-side part. It runs locally on the user's machine, but is designed
      to be easily deployed as an online web application
    \item The editor has several useful features, such as basic support for text
      editing built on top of the CodeMirror JavaScript component with custom
      syntax highlighting and integration with the editor. The text editing
      component is integrated with the visual editing component, so that
      navigation or changes to each representation are tracked and visible to
      the user [[]]
    \item [[]]
\end{itemize}

The language was not originally intended as a Lisp-like language of clone
thereof, but throughout the research I ended up reinventing some language
constructs characteristic of Lisp and learning a lot of and about the language.

An somewhat philosophical interpretation of this would be that Lisp is built on
some fundamental principles that are (re)discoverable rather than invented.



Provided brief specification and description primarily describes the
implementation provided with this thesis. But it is also annotated with
suggestions for improvements or, in other words, descriptions of the more
refined, future design, which itself is subject to change.\js{Nie zrozumiałem
  przesłania tego akapitu.}

\section{Grammar and syntactic features}
Among the main design goals for the prototype of the language were simplicity
and clarity. I wanted a language that is easy to parse and transform to a
different representation. This restriction suggests that the syntax should be as
minimal as possible. A language with one of the most (if not \textit{the} most)
minimal syntax is Lisp\cite{syntaxation}.

An interpreter for Lisp is also trivial to implement, so this is a good starting
point.

There are many approaches to implementing interpreters for LISP in
JavaScript\cite{js_lisps}, but
the general principles are the same.

Although I opted for a minimal syntax, I did not want it to be exactly
Lisp-like, as I thought the syntax of this language could be considerably
improved -- in terms of ease of use and parsing by a human -- with a few simple
adjustments, without significantly increasing the complexity of the interpreter.

There primary criticisms of Lisp's syntax are:
\begin{itemize}
    \item Almost absolute uniformity of syntax makes the source code difficult
      to read by a human and thus [[]]
    \item In general it is hard to teach\cite{wadler_critique}, because complex
      code gets easily confusing
    \item The more nested the syntax tree, the harder it is to keep track of and
      balance parentheses; there tends to be a lot of closing parentheses next
      to each other in the
      source\cite{c2_parentheses}
\end{itemize}

\subsection{Basic syntax}
Before the primary notation of Lisp, namely
S-expressions,
was established -- a slightly different and, in my opinion, slightly more
readable notation was used in the early theoretical publications about the
language, called meta-expressions or M-expressions\cite[Section~The implementation of LISP]{lisp_draft}. I first
simplified this notation in the following way:\js{Należałoby opisać notację
  przed przystąpieniem do opisu jak zostałą ona zmieniona.}

\js{Mam tutaj w ogóle taką myśl, że popełnia Pan tutaj pewien błąd w sposobie
  prezentacji materiału.  Praca naukowa nie powinna stanowić opisu krok po kroku
  jak doszło się do wyniku.  Tutaj mam wrażenie że właśnie coś takiego jest:
  opisuje Pan z czego Pan wyszedł, jakich zmian i dlaczego dokonał, i dopiero na
  końcu przedstawia rezultat.  Moim zdaniem należałoby przedstawić to co Pan
  wymyślił (rezultat), i ewentualnie dopiero potem umotywować czemu tak, a nie
  inaczej.}

\begin{itemize}
    \item I dropped the semicolon \texttt{;} as a separator for arguments, as it
      is entirely superfluous and there is no need for the programmer to type it
      or the parser to be concerned with it; this means that the only separating
      characters are the whitespace characters, exactly as in pure S-expressions
    \item The primary bracketing characters are square brackets (\texttt{[]})
      instead of parentheses; the reason for that design choice is that these
      are easier to type than parentheses or curly brackets (as they do not
      require holding the shift key), which matters considering the ubiquity of
      these characters in the source code
    \item Expression's operator name is written before the opening bracket that
      precedes the list of arguments, as in \texttt{operator[argument-1
          argument-2 ... argument-n]}
    \item I decided not to include any other syntax in the first prototype, as
      the one described so far is entirely sufficient to represent any Lisp-like
      expressions -- it maps directly to S-expressions\footnote{Any additions
        can be considered syntax sugar. Such syntax extensions were introduced
        in later prototypes and designs and some of them are included in the
        final prototype included with this thesis; see [][]}
\end{itemize}

This gives a notation that is somewhat in between S-expressions and
M-expressions. This was the basic syntax of the first prototype of the
language. It can be defined\footnote{This definition is included here only for
  the sake of formality. I believe that for such a simple grammar BNF seems to
  introduce more noise and is unnecessarily more complex than a textual
  description in terms of regular expressions or simply verbatim parser source
  code. For these reasons any extensions to this basic grammar will later on be
  described in these ways.}, using pure, left-recursion-free BNF notation with
the addition of a regular expression (between \texttt{/} delimiters) in the
definition of \texttt{<word>}, as follows:
\begin{lstlisting}
    <expression> ::= <word> | <call>
    <call> ::= <operator> <argument-list>
    <operator> ::= <word> <argument-lists>
    <argument-list> ::= "[" <arguments> "]"
    <word> ::= /[^\s\[\]]+/
    <argument-lists> ::= <argument-list> <argument-lists> | ""
    <arguments> ::= <expression> <arguments> | ""
\end{lstlisting}

The regular expression can be read as ``any character which is not whitespace,
\texttt{[} or \texttt{]}''. This means that aside from whitespace, which acts as
expression separator there are only two special characters -- the square
brackets -- that the parser have to worry about. For this reason it is very
trivial to implement.

The above -- somewhat verbose -- definition is obviously somewhat similar to a
Lisp BNF
description\cite{lisp_bnf_1, lisp_bnf_2}.

An example of a valid expression in light of this definition would be:
\begin{lstlisting}
    do [ bind [a 3] bind [b 5] bind [is-a-greater if [>[a b] true false]]
      ia-greater ]
\end{lstlisting}

Where\footnote{Note: I will introduce brief definitions of language constructs
  as they appear in the presented listings. For a comprehensive list of
  primitives and functions see Section \ref{sec:primitives} of this chapter.}:
\begin{itemize}
    \item \texttt{do} is a language primitive that serves the role of a single
      block of code, much like blocks delimited by \texttt{\{} and \texttt{\}}
      in C-like languages.
    \item \texttt{bind} is a basic construct for defining variables, like
      \texttt{var} or \texttt{define} in other languages.
    \item \texttt{if} serves as a basic conditional evaluation construct. Its
      semantics are like those of the analogous construct in Lisp.
\end{itemize}

Advantages of this M-expression-based notation over S-expressions are:
\begin{itemize}
    \item Easier to parse by a human. Operators are clearer distinguished from
      operands. This is arguably because this notation is more familiar, bearing
      a similarity to the general mathematical notation (as in \texttt{f(x)})
      and the most popular programming language syntax -- the C-like
      syntax\footnote{11 out of the
        top 20 languages as of June 2016\cite{tiobe} have C-based syntax (by this
        classification:
        \cite{c_family_list_wikipedia}). If
        we extend the syntax family to Algol-like, its virtually 20 out of 20 --
        \cite{pl_genealogy}. There
        are no languages with Lisp-based syntax among the most popular ones.}
    \item If an expression has another expression as its operator, it is written
      as \texttt{op[args-1][args-2]}, which reduces the amount of nesting and
      thus the amount of bracketing characters appearing next to each other in
      the source code. Compare the equivalent S-expression: \texttt{((op args-1)
        args-2)}; and with multiple levels:
      \texttt{op[args-1][args-2][args-3][args-4]} vs \texttt{((((op args-1)
        args-2) args-3) args-4)}
\end{itemize}

An interesting property of this syntax, that, depending on the context could be
classified as advantage, disadvantage or neither is that the sequence of
characters \texttt{[[} is not legal, whereas in Lisp the analogous sequence
    \texttt{((} is.

Alas, this simple notation doesn't do away with a lot of other problems inherent
in any minimal syntax, because such syntaxes have the property of being very
homogenous. In the next subsections and later throughout this thesis I gradually
introduce extensions, which make the syntax a little bit more diverse. Keep in
mind that every special character that is introduced, is taken away from the set
of possible \texttt{<word>}-characters, which implies that the regular
expression for \texttt{<word>} is changed accordingly.

\subsection{Comments, whitespace and the syntax tree}\label{sub:comments}
\js{Rozdział o komentarzach zajmuje więcej miejsca niż opis z opisem składni!}

The parser was further extended with support for comment syntax similar to the
ones found in Ada, Haskell or Lua:
\begin{lstlisting}
    -- a comment that extends until the end of the line an expression that
    -- computes square root of 81: sqrt[81]
    
    --[ this is a multiline comment
        
        --[ multiline comments can be nested
            
            as long as [ and ] are balanced, anything can be nested within
            multiline comments
            
            for example: --[this is a comment that includes a piece of code *[7
                7], which would evaluate to 49] ] ]
\end{lstlisting}

In principle multiline comments could be implemented simply with the syntax
analyzer checking the operator of the expression being parsed, and if it is
\texttt{--}, treating such expression as a comment. The fact that this
expression was already parsed and transformed into a structural tree-like form
could be taken advantage of while generating documentation from comments. For
example we could define a following \acrlong{dsl}\footnote{Inspired by
  \cite{jsdoc_wikipedia})} for documentation:
\begin{lstlisting}
    --[ the below is a documentation comment followed by the documented piece of
        -- code: [[ Calculates the circumference of the Circle.
            
            override!  deprecated!
            
            this [circle]
            
            -- The circumference of the circle: return [number] --]]
        
        define [calculate-circumference procedure [ mul[2 math.pi this.radius]
    ]] ]
\end{lstlisting}

Nevertheless the implementation in the prototype treats the comments as a stream
of characters, taking into account nesting and balancing of brackets, but
doesn't keep them as a tree-like structure.

Whitespace characters and comments have no semantic significance, unless serving
as separators could be considered one. After building the syntax tree,
bracketing characters also serve no purpose and can be safely be discarded,
without influencing the interpretation of the program.

Despite this, all of these are included in the syntax tree generated by the
parser. Storing these characters in the syntax tree means that the entirety of
the textual representation, in structural form, is accessible by any other
representation we might devise. This allows for implementation of variety of
interesting ``smart'' language and editor features.

A thing to note at this point is that such syntax tree can't be described as
``abstract''\footnote{According to these definitions:
  \cite{ast_wikipedia, c2_ast}}, as it includes all of the
``concrete'' syntax, with its runtime-insignificant elements. The syntax tree
that is built for the Dual language is also later extended with references to
other representations of code. For these reasons, I will later on use the
acronym \acrlong{est} to refer to the representation used internally by the Dual
language interpreter.

This representation greatly simplifies the implementation of and integrates with
the language the following features:
\begin{itemize}
    \item Automatic indentation
    \item Documentation comments. Comments can easily be associated with
      corresponding code blocks (syntax tree nodes), which can be useful for
      automatically generating documentation in any format
    \item Any expression can be unparsed to its original form straight from
      syntax tree, which can be used for debugging
    \item This also means that any expression can be stringified on-the-fly and
      this string can be used as a value in the program. This feature allowed me
      to completely omit definition of strings at the parser level, although
      this is not a very efficient solution. Nevertheless keeping strings in
      such structural form -- as syntax trees -- in combination with pattern
      matching enables language-native structural manipulation of
      strings\footnote{See:
        \cite{wolfram_string_patterns}
        and
        \cite[Section~Pattern matching and strings]{pattern_matching_wikipedia}
        for similar concepts.}. For example
      we could write:
    \begin{lstlisting}
        bind [str '[A quick brown fox jumps over the lazy dog]]
                    
        bind [words [_ _ third-word {rest}] str]
        
        bind [characters [_ _ third-letter {rest}] third-word]
        
        -- logs "o" to the console: log [third-letter]
    \end{lstlisting}
    
Where \texttt{words} deconstructs a string into single words and binds these
words to identifiers provided as its arguments and \texttt{characters} performs
an analogous operation on the single character-level. The notation
\texttt{\{rest\}} matches zero or more arguments (see Section \ref{sub:rest} for
details). \texttt{log} outputs the values of its arguments to the JavaScript
console.

Note that I chose the structural representation as
the only representation for strings, because the performance penalty is acceptable in the prototype implementation. An obvious and very simple optimization
would be to keep the raw form of the string as a value in the
corresponding syntax tree node. Having these two forms alongside each
other would enable the programmer to use the familiar string
manipulation methods as well as structural manipulation.
\end{itemize}

Such representation was implemented in order to fulfill the design goal of
direct and complete mapping of the textual representation into any other
representation.

\subsection{Numbers}
Numbers in the language are represented as JavaScript numbers. This means that
there's only one number type -- 64-bit floating point\footnote{Defined by the
  ISO/IEC/IEEE 60559:2011 (IEEE 754) standard:
  \cite{60559_2011, js_numbers}}. They are
implemented as follows:
\begin{itemize}
    \item When a word is tokenized by the parser, it is converted to a
      JavaScript number with a Number type constructor, which returns either the
      corresponding value (if the word is parsable to a number) or the value
      \texttt{NaN}. In the former case, the numerical value is stored in the
      appropriate syntax tree node, as its \texttt{value} property.
    \item Upon evaluation, a syntax tree node is checked for the \texttt{value}
      property. If it has one it is given as the result of the evaluation.
    \item The fact that a number is stored as a syntax tree node, which contains
      the its string representation and its raw value, both obtained from the
      source code during parsing means that conversion from a number literal to
      string is zero-cost, which could be useful for optimization.
\end{itemize}

This shows a possible way of optimizing the representation of strings, which I
intend to introduce in a future version of the language.

\subsection{Zero and single argument expressions}
In order to reduce the amount of \textit{closing} brackets appearing next to
each other in program's text, two additional simple notations were
introduced. The first is addition of the pipe special character
(\texttt{|}). This character is used for single-argument expressions, as in:
\js{Najpierw należałoby wyjaśnić jak to działa, a dopiero potem dawać przykłady.
  Swoją drogą warto napisac że | wiąże od prawej, tzn. foo | bar | baz to to
  samo co foo [bar [baz]]}

\js{jestem za dopracowaniem wyglądu listingów - czcionka o stałej szerokości
  mile widziana.}
\begin{lstlisting}
    -- compute factorial of 32: factorial|32 -- equivalent to factorial[32]
    
    -- find 9th Fibonacci number: fibonacci|9 -- <=> fibonacci[9]
    
    -- compute sine of pi sin|pi -- <=> sin[pi]
    
    -- compute cosine of the number that is the result of multiplication of pi
    -- and 5!: cos|*[pi factorial|5] -- <=> cos[*[pi factorial[5]]]
    
    -- convert 33.2 to an integer (truncate .2): to-int|33.2 -- <=> to-int[33.2]
    
    -- construct a list with one item, which is a string "hello" list|'|hello --
    -- <=> list['[hello]]
\end{lstlisting}

The above example shows that if a function is invoked with one argument, we can
omit the closing brace and replace the opening brace with \texttt{|}. The parser
produces equivalent syntax tree.

Another special character (\texttt{!}) was introduced for analogous use for
zero-argument expressions (procedures):
\begin{lstlisting}
    -- invoke a procedure that changes some state variables in its outer scope:
    -- set-initial-state! -- <=> set-initial-state[]
    
    -- sum two random numbers: <=> +[random[] random[]]: +[random! random!]
    
    -- bind a value returned by an immediately invoked procedure to an
    -- identifier <=> bind [forty-two procedure [42][]] bind [forty-two
    -- procedure [42]!]  forty-two -- evaluates to 42
\end{lstlisting}

\subsubsection{In combination with macros}\label{subsub:macros}
A unique macro system, described in Chapter \ref{chap:design}\footnote{This
  macro system is not included in the final version of the prototype, although a
  proof-of-concept of it that I implemented in earlier prototypes is the basis
  for this description. \js{Dobrą strategią w takich przypadkach jest zrobienie
    rozdziału opisującego projekt teoretyczny, a następnie rozdziału z opisem co
    rzeczywoiście zostało zaimplementowane.  Pan ma podobnie, ale jednak pisze
    Pan co zostało a co nie zostało zaimplementowane już tutaj.}} in combination
with these two (\texttt{|} and \texttt{!}) special characters help reduce the
amount of bracketing characters even further.\js{Nie podoba mi się, że żeby
  zrozumieć co tu jest napisane muszę zajrzeć do dalszego rozdziału.}

For example, if we define a \texttt{match*} and \texttt{of*} macros as
described, the following expression:
\begin{lstlisting}
    bind [x 99]
    
    -- will log "x is greater than one": match* [x] | of* [<|0] [log|'[x is
        negative]] | of* [0] [log|'[x is zero]] | of* [1] [log|'[x is one]] |
    log|'[x is greater than one]
\end{lstlisting}

which is somewhat similar syntactically to
ML-style\cite[Section~Algebraic datatypes and pattern matching]{standard_ml_wikipedia}
languages, could be translated into the following:
\begin{lstlisting}
    bind [x 99]
    
    -- will log "x is greater than one": apply [ of [<|0 log|'[x is negative] of
        [0 log|'[x is zero] of [1 log|'[x is one] log|'[x is greater than one] ]
      ] ] x ]
\end{lstlisting}

where \texttt{apply} would be defined analogously to Lisp's
\texttt{apply}. \texttt{of} would be defined as a function primitive with arity
2..*, which treats its penultimate argument as the function's body and all the
preceding arguments as patterns for the function's arguments. The last argument
is used when such a function is called and the values supplied as arguments
don't match the patterns. If the argument is a function, it will be called with
the same values as arguments and if it's a value it will be returned.

I used such a solution for pattern matching in the early prototypes, but
replaced it with a native \texttt{match} construct (described in Section
\ref{sec:primitives}) for performance reasons. Nevertheless this shows that a
few simple, but general syntax rules and a powerful macro system, can be a very
flexible tool for extending syntax.

The pattern matching mechanism is explained in the next subsection.

\subsection{Pattern matching}
A simple, yet powerful pattern-matching facility was added to the language.

Pattern matching works with bindings, functions (although the primary
function-producing expression in the prototype doesn't use it by default),
\texttt{match} primitive and macros (not available in the prototype).

The pattern matching works in a way similar to most other languages that support
this feature (e.g. ML family). The general rules are\footnote{For brevity I
  assume here that `of` is a primitive that works like described in
  \ref{subsub:macros}, where the alternative argument is optional. This is how
  it was implemented in an early prototype of the language. If no viable
  alternative was present, an error was thrown.}:
\begin{itemize}
    \item A literal (strings or numbers are supported) value matches itself:
    \begin{lstlisting}
        -- computes factorial of a number bind [factorial of [0 1 of [n *[n
                factorial[-[n 1]]]]] ]
        
        -- logs `120`: log [factorial|5]
    \end{lstlisting}
    \item An identifier (word) matches any value, which is then bound to the
      identifier:
    \begin{lstlisting}
        bind [simple-print of [x log|x]]
        
        -- logs `3`: simple-print[3]
    \end{lstlisting}
    \item A wildcard pattern (\texttt{\_}) matches any value, but doesn't bind:
    \begin{lstlisting}
        -- returns its third argument, discards the rest: bind [get-third of [_
        -- _ x x]]
        
        -- logs `3`: log [get-third[1 2 3]]
    \end{lstlisting}
    As such it can be useful for discarding some values, depending on other
    values or extracting some values from a structure (see next point).
    \item The following expression-patterns are supported:
    \begin{itemize}
        \item \texttt{list} or \texttt{\$} is used to destructure lists:
        \begin{lstlisting}
            bind [$[_ _ third-element] $[0 1 2]]
            
            -- logs `3` log [third-element]
            
            -- it works for arbitrarily nested lists as well bind [ $[ _ $[ _
            -- pick _ _] _] $['|a $['|b '|c '|d '|e] '|f] ]
            
            -- logs `c`: log [pick]
        \end{lstlisting}
        \item Comparison operators (\texttt{= < <= >= <>}) match if a value
          passes the comparison; it can be viewed as a shorthand notation for
          simple
          guards\cite[Chapter~Pattern Matching Basics, Section Using Guards within Patterns]{f_sharp_wikibooks}:
        \begin{lstlisting}
            -- returns the sign of a number note: `-#` is the unary `-`
            -- operator: bind [sign of [=|0 0 of [<|0 -#|1 of [>|0 1]]] ]
            
            -- logs `-1`: log [sign|-77]
        \end{lstlisting}
        \item Other pattern-expressions are not supported and using them will
          result in a mismatch.
    \end{itemize}
\end{itemize}

The above examples show pattern matching used for destructuring values and
binding their components to identifiers and for function definitions. There's
also a \texttt{match} primitive, which can serve the role of a \texttt{switch}
statement from C-like languages. Although pattern matching makes it much more
powerful than that, as any values supported by the pattern matching system can
be matched, including lists, which allow us to switch on multiple values and in
any combinations.

The \texttt{match} primitive's first argument is a value to match and all
subsequent arguments are two-element lists, where the firs element is the
pattern to match and the second is the expression to evaluate in case of a
match. The primitive tries the matches in order and only evaluates the
expression, related to the successful match, which is the first one that
matches. The subsequent matches are not evaluated.

\begin{lstlisting}
    bind [state '|game-on]
    
    -- will execute the `play` procedure: match [state $['|game-on play!]
      $['|game-paused display-pause-menu!]  $['|game-screenshot
        capture-screenshot!]  ]
    
    -- ...

    -- note: . is the access operator .[a b c] is equivalent to a.b.c in other
    -- languages bind [$[x y] .[player postion]]
    
    -- we can easily replace complex conditions: match [$[x x y y] $[ $[>|0
          <|screen-width >|0 <|screen-height] log|'[player visible] ] $[_
        log|'[player not visible]] ]
    
\end{lstlisting}
\begin{itemize}
    \item 
\end{itemize}

With an arsenal of these few simple pattern matching tools we can use a lot of
useful features, which further add expressivity to the language. We can also
imagine many possible extensions and generalizations, as briefly discussed in
Chapter \ref{chap:design}.

\begin{itemize}
    \item Destructuring assignments or, more precisely, destructuring
      definitions.\footnote{Destructuring could easily be extended to mutation
        as well, although I have found it sufficient to be usable only in
        definitions, while implementing the prototype.}. An example of such
      definition would be:
    \begin{lstlisting}
        bind [$[a b $[c d]] $[1 2 $[3 4]] bind [ $[ _ x y { rest }] $['|a '|b
              '|c '|d '|e '|f] ]
        
        -- logs `1 2 3 4`: log [a b c d]
        
        -- logs `b c ["d", "e", "f"]`: log [x y rest]
    \end{lstlisting}
\end{itemize}


\subsection{Rest parameters and spread operator}\label{sub:rest}
\js{Przyznaję, ze czytałem szybko, ale nie zrozumiałem do końca tych operatorów.}
Another syntax extension that I introduced involved two additional special
bracketing characters: \texttt{\{} and \texttt{\}}, which serve several
purposes:
\begin{itemize}
    \item Rest parameters mechanism known from
      Lisp\cite[Section~12.2.3]{emacs_lisp_reference},
      recently also adopted in JavaScript (as of the ECMAScript2015
      standard\cite{mdn_rest}). That
      is, for example:
    \begin{lstlisting}
        bind [variadic-function of [a b {args} log [a b args] ]]
        
        -- logs `1 2 [3, 4, 5, 6]`: variadic-function[1 2 3 4 5 6]
    \end{lstlisting}
    This enables the user to easily define variadic functions, which can be
    called with a variable number of arguments.  This works in any place, where
    pattern matching works:
    \begin{lstlisting}
        bind [$[a b {rest}] $['|a '|b '|c '|d '|e]]
        
        -- logs `["c", "d", "e"]` log [rest]
    \end{lstlisting}
    
    thus enabling non-exact matching.
    
    \item Spread operator (also inspired by the analogous feature from
      ECMAScript2015):
    \begin{lstlisting}
        bind [f of [a b c d e f log [a b c d e f]]] bind [args $[8 7 6]]
        
        -- logs `9 8 7 6 5 4`: f[9 {args} {$[5 4]}]
    \end{lstlisting}
    This provides a much nicer and more powerful alternative to \texttt{apply},
    Lisp's fundamental function, which applies a function to a list of
    arguments. It's a way to flatten any list onto a list of arguments. This
    works for multiple values and lists as well:
    \begin{lstlisting}
        -- alternative way to achieve the same result as in the previous listing
        -- logs `9 8 7 6 5 4`: f[{9 args $[5 4]}]
    \end{lstlisting}
    \item String interpolation notation:
    \begin{lstlisting}
        bind [name '|Bill]
        
        -- logs `Hello, Bill.` log ['[Hello, {name}.]]
    \end{lstlisting}
    
    As we can see this gives us a very convenient notation for string
    interpolation, similar to e.g. template literals in
    JavaScript\cite{mdn_template_strings}.
    In order to escape curly braces, they should be doubled:
    \begin{lstlisting}
        -- logs `Hello, {name}.` log ['[Hello, {{name}}.]]
    \end{lstlisting}
    
    I also added a special type of string -- an HTML string, where interpolation
    notation is the other way around -- double braces cause substitution, single
    braces do nothing:
    \begin{lstlisting}
        bind [name '|Bill] -- logs `<h1>Hello, Bill.</h1>` log [html'[<h1>Hello,
            {{name}}.</h1>]]
        
        -- logs `<h1>Hello, {name}.</h1>` log [html'[<h1>Hello, {name}.</h1>]]
    \end{lstlisting}
    
    This is to enable embedding CSS and JavaScript code inside those strings,
    without having to constantly escape brace characters.
    
    \item Unquote notation for macros\footnote{Analogous to \texttt{unquote} or
      \texttt{,} in Lisp:
      \cite[Section~1.3.8]{racket_reference}
      -- see Chapter \ref{chap:design}}.
\end{itemize}

\section{Basic primitives and functions}\label{sec:primitives}
Below is a description of the primitives and functions supported by the Dual
language. Each item is structured as follows:
\begin{itemize}
    \item \texttt{<name> [<arguments>]}
    
    <description>
    
    Where \texttt{<name>} is the name of the function/primitive and
    \texttt{<arguments>} are either the names that describe the arguments of the
    function/primitive or its arity. That is, the number of arguments that the
    function/primitive is defined for. This can be a fixed value (e.g
    \texttt{1}), a fixed range of values (e.g. \texttt{0..3}) or a range of
    values without an upper bound (e.g. \texttt{0..*}, which means 0 or more).
    
    <description> is a brief description of the function/primitive.
\end{itemize}

\subsection{Language primitives}
\js{Wydaje mi się, że nalezałoby to opisać na samym początku. Czyli najpierw
  ogólna składina hęzyka, potem operatory prymitywne, pattern matching, i potem
  reszta. Komentarze na końcu.}  The Dual language supports the following
primitives:
\begin{itemize}
    \item \texttt{do [0..*]}
    
    Evaluates its arguments in order and returns the value of the last argument.
    
    \item \texttt{bind [name value]}
    
    Evaluates its second argument and binds this value to the name of the first
    argument. This name is bound within the current scope. This is a basic
    construct for defining variables, like \texttt{var} or \texttt{define} in
    other languages. Significant semantics here are that new scopes are
    introduced by function bodies, macro bodies and match expression bodies. The
    primitive also supports pattern matching to deconstruct the value and bind
    its components to possibly several variables. In that regard it works a lot
    like JavaScript's destructuring
    assignment\cite{mdn_destructuring}
    or similar features in other languages, such as Perl or Python. This
    primitive can be used only for binding names that don't exist in the scope
    at the point of its invocation. There are other constructs for mutating and
    modifying existing variables. There is no
    hoisting\cite[Section~var hoisting]{mdn_var},
    as definitions are processed in order in which they appear in code.
    
    \item \texttt{if [condition consequent alternative]}
    
    This primitive serves as a basic conditional evaluation construct. Its
    semantics are like those of the analogous construct in Lisp. It accepts 3
    arguments: first the \texttt{condition} expression, then the
    \texttt{consequent}, that is, the expression to be evaluated if the value of
    the condition is \textit{not false} (note that this is a strict rule; any
    other value than \texttt{false} is interpreted as \texttt{true}; every
    conditional construct in the language follows this rule). The third
    argument, the \texttt{alternative} is the expression that is evaluated
    otherwise.
    
    \item \texttt{while [condition body]}
    
    A basic loop construct. If \texttt{condition} is equivalent to \textit{not
      false}, evaluates \texttt{body}. Repeats these steps until
    \texttt{condition} evaluates to \texttt{false}. Returns the value of the
    last evaluation of \texttt{body} or \texttt{false} if the body was not
    evaluated.
    
    \item \texttt{mutate* [name value]}
    
    If a variable identified by \texttt{name} is defined within the current
    scope or any outer scope, changes (mutates) its value, so it now refers to
    the result of evaluating the \texttt{value} argument. The scopes are
    searched from the innermost to the outermost, in order. If the \texttt{name}
    argument doesn't identify any variable, an error is thrown. Returns the
    scope (environment), in which the primitive was evaluated. [[first-class
        environments, Bla paper?]]
    
    \item assign
    
    \item code
    
    \item macro
    
    \item of
    
    \item of-p
    
    \item procedure
    
    \item match
    
    \item cons
    
    \item invoke*
    
    \item .

    \item :
    
    \item @
    
    \item dict*
    
    \item async*
\end{itemize}

Basic functions and values:
\begin{itemize}
    \item \texttt{true} and \texttt{false} evaluate to their respective boolean
      values. \texttt{\_} is an alias for \texttt{true} when used outside of
      pattern-matching. This enables a convenient compatibility between
      \texttt{match} and \texttt{cond}: if we're matching a single value and
      want to have a default case, then \texttt{\_} is used to match any
      value. Similarly, if \texttt{\_} is given as a condition in the last
      alternative of \texttt{cond}, it will evaluate to \texttt{true} and work
      as the default case.
    \item \texttt{undefined} evaluates to JavaScript's undefined[[link]].
    \item \texttt{typeof} wraps JavaScript's \texttt{typeof} operator.
    \item \texttt{or} and \texttt{and} are the basic logical operators --
      analogous to \texttt{||} and \texttt{\&\&} in JavaScript.
    \item \texttt{any} and \texttt{all} are like the above, but accept variable
      number of arguments. These return either \texttt{true} or \texttt{false}.
    \item \texttt{not} is the negation operator (\texttt{!})
\end{itemize}

%
% Design discussion
%

\section{Comments}\label{sec:comments}
If multiline comments were implemented as expressions on parser-level then, in
combination with \texttt{|} special character we could have one-word comments,
which could be useful for describing arguments to facilitate reading of
expressions. For example we could implement list comprehensions, where:
\begin{lstlisting}
$<-[^[x 2] x range[0 10]] $<-[$[x y] x $[1 2 3] y $[3 1 4] <>[x y]]
\end{lstlisting}
would be equivalent to
Python's\cite[Section~5.1.3]{python_tutorial}:
\begin{lstlisting}
[x**2 for x in range(10)] [(x, y) for x in [1,2,3] for y in [3,1,4] if x !=
y]
\end{lstlisting}
As we see this notation is acceptable (if not cleaner) for simple
comprehensions, but starts being less readable for complex ones. This could be
alleviated by introducing one-word comments:
\begin{lstlisting}
$<-[^[x 2] --|for x --|in range[0 10]]

$<-[$[x y] --|for x --|in $[1 2 3] --|for y --|in $[3 1 4] --|if <>[x y]]
\end{lstlisting}
which are easily inserted inline with code and have a benefit of clearly
separating individual parts of an expression, because of being easily
distinguished visually from the rest. This can simulate different syntactical
constructs from other programming languages, like:
\begin{lstlisting}
if [>[a b] --|then log['|greater] --|else log['|lesser-or-equal] ]
\end{lstlisting}
Except that it is not validated by the parser. But we could imagine a separate
or extend the existing syntax analyzer, so it could validate such ``keyword''
comments or even use them in some way. For example, we could add a static type
checker to the language -- in a similar manner that TypeScript or
Flow\cite{flow_site} extends JavaScript. This would be
completely transparent to the rest of the language, so any program that uses
this feature would be valid without it and it could be turned on and off as
needed.

To reduce the number of characters that have to be typed, we could decide to use
a different comment ``operator'', such as \texttt{\%}:
\begin{lstlisting}
$<-[^[x 2] %|for x %|in range[0 10]]

$<-[$[x y] %|for x %|in $[1 2 3] %|for y %|in $[3 1 4] %|if <>[x y]]

if [>[a b] %|then log['|greater] %|else log['|lesser-or-equal] ]
\end{lstlisting}

Or even, at the cost of complicating the parser, introduce a separate syntax for
one-word comments:
\begin{lstlisting}
-- `%:type` could be a type annotation
bind [a 3 %:integer]
bind [b 5 %:integer]

-- will print "lesser-or-equal"
if [>[a b] %then
log['|greater]
%else
log['|lesser-or-equal]
]
\end{lstlisting}

In future versions of the language, comments will be stored separately from
whitespace in the EST. This enables easy smart indentation -- only a prefix of
the relevant expression has to be looked at, no need to filter out comments. It
also enables using comments structurally, as a metalanguage for annotations,
documentation, etc.

\section{Structural string manipulation}
\begin{lstlisting}
words[_ _ _ fourth _ sixth] -- super fast, out of the box characters[_ _ _ _
fifth]
\end{lstlisting}

\section{Just-in-time macros}
One feature that I experimented with while creating the prototype of Dual is
support for ``first-class just-in-time expanded macros''. By this I mean macros
similar to those found in Lisp, but with a few key characteristics, which
differentiate them from conventional implementations of macros in Lisp-like
languages. These are:
\begin{itemize}
    \item Macros are expanded upon evaluation; when a macro invocation is
    encountered by the interpreter, it is expanded into code; the node in
    the EST containing the macro invocation is permanently replaced by the
    expanded expression, which is subsequently evaluated and its value is
    returned as the value of the invocation.
    \item Macros can return other macros in a straightforward manner; this
    feature nicely composes with the variation of Lisp syntax found in
    Dual; I used it extensively to improve it, mostly with the goal of
    reducing the amount of adjacent closing brackets in the source code.
\end{itemize}

In order to support first-class runtime macros a Lisp interpreter can be
modified as follows\cite{macros}:
\begin{itemize}
    \item Primitives are moved into the top-level environment; they thus are
    no longer treated as special case by the \texttt{eval} function.
    \item A new primitive, \texttt{macro} is added, which is essentially
    equivalent to \texttt{lambda}, except that it produces macro values
    instead of function values.
    \item The \texttt{apply} function is now responsible for checking the
    type of an expression's operator, which can be a \texttt{primitive}, a
    \texttt{macro} or a normal expression; this determines whether the
    arguments are evaluated before application
\end{itemize}

This results in a simpler, more uniform and at the same time more powerful
interpreter. A major advantage is that:
\begin{quote}
    Because of their first-class nature, first-class macros make it easy to add or
    simulate any degree of laziness\cite{macros}
\end{quote}

The below listing presents an example of a macro named \texttt{if*} that wraps
the \texttt{if} primitive in a slightly different syntax. This syntax wraps the
condition, consequent and alternative parts of the \texttt{if} in separate
blocks delimited by \texttt{[]}. The condition is required to be an infix
expression in the form \texttt{a operator b}. The consequent and alternative
blocks take care of wrapping all expressions within them in \texttt{do}
blocks. This makes it more convenient and less error-prone to write complex
\texttt{if} expressions:
\begin{lstlisting}
bind [if* macro [a op b macro [{then} macro [{else} code'[if [apply[{op} {a}
{b}] do[{then}] do[{else}]]] ]]]

if* [a < b][ log ['[a is less than b]] a ][ log ['[b is less than or equal to
a]] b ]

-- expands to: if [<[a b] do [ log ['[a is less than b]] a ] do [ log ['[b is
less than or equal to a]] b ]]
\end{lstlisting}

TODO: elaborate

***

A somewhat tangent, but interesting observation here is that a Lisp interpreter
could be simplified further by removing all special cases from
\texttt{apply}. It would only apply a function to arguments, without checking
its type. This would cause the following:
\begin{itemize}
    \item All primitives would cease to be ``special''.
    \item If we keep the strict evaluation strategy, a programmer would have
    to explicitly quote all expressions that should not be immediately
    evaluated.
    \item We could also switch to a variation of lazy evaluation, which
    could be best described as ``explicit evaluation'', where evaluation
    \textit{never} happens unless explicitly requested. Under this
    evaluation strategy, a programmer would have to explicitly evaluate
    all expressions that should be evaluated.
\end{itemize}

Assuming the explicit evaluation strategy, the \texttt{if} primitive could be
defined in the interpreter as follows (in JavaScript):
\begin{lstlisting}
// args is the list of arguments, which would be provided by apply // this is a
greatly simplified implementation, to demonstrate the essence of the idea:
function if (args, environment) { var condition = args[0], consequent = args[1],
alternative = args[2]; var conditionValue = evaluate(condition, environment);

if (conditionValue) { return evaluate(consequent, environment); } return
evaluate(alternative, environment); }
\end{lstlisting}

Assuming we have \texttt{bind} and all other necessary primitives defined to
conform to this evaluation strategy and a terse syntax for explicit evaluation
(\texttt{'}):
\begin{lstlisting}
-- ' causes an expression to be evaluated

-- bind would always evaluate its second argument: 'bind [a 5]

-- +, - and other arithmetic operators would always evaluate all of their
arguments: 'bind [b +[a 7]]

-- log would evaluate all of its arguments logs `12`: 'log [b]

-- of would not evaluate any of its arguments: 'bind [factorial of [n do [ 'bind
[n-value n] *[n-value factorial[-[n-value 1]]] ]]]
\end{lstlisting}

TODO: how this could be useful compiling to simplified Lisp

\section{C-like syntax}
Throughout this thesis I introduced multiple ways in which the basic, Lisp-like
syntax of Dual can be easily extended with simple enhancements, such as adding
more general-purpose special characters, macros, single-word comments (as
described in Section \ref{sec:comments}), etc.

Going further along this path, keeping in mind that a real-world language should
appeal to its users we find ourselves introducing more and more elements of
C-like syntax. This section describes more possible ways in which the simple
syntax could be morphed to resemble the most popular languages.  Ultimately all
this could be implemented with a conventional complex parser for a C-like
language that translates to bare Dual syntax.

Below I present a snapshot from one of designs I have been working on in order
to achieve some goals described in this section:
\begin{lstlisting}
fit map" {f; lst} { let {i; ret} [0, []];

while ((i < lst.length)) { ret.push f(lst i); set i" ((i + 1)) }; ret };
\end{lstlisting}

This would be equivalent to:
\begin{lstlisting}
bind ['|map of ['|f '|lst do [ bind ['[i ret] $[0 $[]]]

while [<[i lst|length] do [ ret[push][f[lst|@[i]]] mutate* ['|i +[i 1]]
]] ret ]]]
\end{lstlisting}

Using the notation presented in Chapter \ref{chap:lang}.

One may observe that:
\begin{itemize}
    \item The syntax is much richer, somewhat C-like, but with critical
    differences, reflecting significantly different nature of the language. At
    a first glance, it has a familiar look defined by blocks of code delimited
    by curly-braces, inside which statements (actually expressions) are
    separated by semicolons; there are different kinds of bracketing
    characters (\texttt{\{\}()[]}) with different meanings (described below)
    \item Names of the primitives are \textit{full} English words, although as
    short as possible. \texttt{let} introduces a variable definition --
    similarly to \texttt{bind}. \texttt{fit <name> <args> <body>} is a
    shorthand for \texttt{let <name> (of <args> <body>)}, where \texttt{of}
    produces a function value. This translates to \texttt{bind [<name> of
        [<args> <body>]]}.
    \item \texttt{\{\}} delimit a string; inside a string words are separated by
    \texttt{;}. Strings are stored in raw as well as structural (syntax tree)
    form. They are a way of quoting code. This provides an explicit laziness
    mechanism. One-word strings are denoted with \texttt{"} at the end of the
    word, which resembles the mathematical double prime notation.
    \item \texttt{[]} delimit list literals; inside list literals, elements are
    separated by \texttt{,}. Lists are a basic data structure. They are
    actually objects, somewhat like in JavaScript. If a list contains at least
    one \texttt{:} character (not shown in the example), it will be validated
    as key-value container; if it doesn't, it will be treated as array with
    integer-based indices
    \item \texttt{()} are used in function invocations; \texttt{f(a, b, c)}
    translates to \texttt{f[a b c]}; \texttt{,} separates function arguments;
    \texttt{f x} is a shorthand notation for \texttt{f(x)}. This, in
    combination with currying primitives into appropriate macros allows for
    elimination of excessive brackets and separators. Invocations of
    primitives resemble use of keywords from other lanugages.
    \item But at the same time primitives are defined as regular functions --
    they are no longer treated exceptionally by the interpreter. When they are
    invoked, all of their arguments are first evaluated. This works, because
    now it is required that the programmer quote any words that shouldn't be
    evaluated, such as identifier names when using \texttt{let}. So primitives
    are just regular functions operating on code, thanks to the explicit
    laziness provided by strings.
    \item \texttt{(())} introduce an infix expression, which respects basic
    operator precedence: (\texttt{((a + b * 2))} would translate to
    \texttt{+[a *[b 2]]}. This could be implemented with a separate parser
    based on the
    shunting-yard\cite{shunting_yard} or
    similar algorithm that is triggered by the \texttt{((} sequence. It would
    translate these infix expressions to prefix form and return them back to
    the original parser.
\end{itemize}

\section{Universal visual editor}
The structure produced by a visual editor does not have to necessarily be a
syntax tree of a particular language.  Text editors allow inputing arbitrary
sequences of characters, which are transformed into a syntax tree by a
specialized parser. Analogously, an universal visual editor could allow
insertion, connection and manipulation of arbitrary generalized blocks, thus
forming a truly abstract syntax tree (language-independent). This tree could
then be ``parsed'' to produce a syntax tree for a particular language.

\section{Lua}
A very interesting programming language, predating and somewhat similar to JavaScript is Lua. It is designed to be a minimalist, embeddable scripting language. Similarly to JavaScript, it draws heavily from Scheme. Some of the major and interesting features of the language include:
\begin{itemize}
    \item First-class functions, closures and lexical scoping.
    \item Tables as the basic data structure, similar to JavaScript's WeakMaps; more powerful than plain JavaScript objects, because they additionally support:
    Non-string keys -- any value can be used as a key, except nil and NaN
    \item Extension through metatables
    \item Excellent interoperability with C
\end{itemize} 


Many computer games use Lua as a scripting language\cite{lua_games_category_wikipedia}
Prototype-based OOP with syntactic sugar for method definitions and calls

There also exists a very efficient implementation of the language called LuaJIT\cite{LuaJIT}. It is actually one of the fastest JIT-compiled language implementations, rivaling or surpassing performance of most other JIT-compiled languages, including modern JavaScript engines and even, in some benchmarks, native-compiled languages such as C++, D or C\cite[Speed vs other languages]{lua_perl}\cite[Section~Results]{mateo_benchmarks}\cite{wren_performance, pl_benchmarks, lua_c_performance}.

Recent version of Lua (5.3)\cite{lua_5_3} introduces support for integer type, basic unicode (UTF-8) support and bitwise operators, which were long-missing features.

It has many features that I

\section{Performance}
JIT-compilation
Bytecode

Problem:
no more web platform

Solution:
WebAssembly

\section{Class-free Object-Oriented Programming}
A major paradigm I intend to support in future versions of Dual is class-free \acrshort{oop}.

In \cite{crockford_class_free} Douglas Crockford states:
\begin{quote}
    I used to think that the important innovation of JavaScript was prototypal inheritance. I now think it is class-free object oriented programming. I think that is JavaScript’s gift to humanity. That is the thing that makes it really interesting, special, and an important language. 
\end{quote}

He defines a constructor template that demonstrates this paradigm. Below is a slightly more verbose and annotated for clarity version of this constructor:
\begin{lstlisting}
function constructor(specification) { // [0]
let { member1, member2 } = specification, // [1]
{ other }  = other_constructor(specification), // [2]
private_method = function () { // [3]
// accesses member, other, method, spec
},
public_method = function () { // [4]
// accesses member, other, method, spec
};

return Object.freeze({ // [5]
public_method,
other
});
}
\end{lstlisting}

This is Crockford's simplification and evolution of prototype-based OOP. It actually gets rid of references to the \texttt{prototype} property as well as any use of the keyword \texttt{this}. It relies on composition with destructuring and copying instead of delegation, prototype chains and sharing of properties.

This has a disadvantage in increased memory consumption, but disables what Crockford calls ``retroactive heredity'', which means changing an object's prototype after it is created, at runtime. But, as he rightly notes, this feature of prototype-based OOP is rather harmful than beneficial. It can create all kinds of errors, can be confusing, has a big performance impact, as it gets in the way of optimization techniques employed by modern JavaScript engines\cite{mdn_set_prototype_of, v8_design}.

This approach to OOP is very flexible. It allows for creation of private, protected (privileged) and public members through standard JavaScript patterns, such as (revealing) module pattern\cite[Chapter~JavaScript Design Patterns, Section~The Revealing Module Pattern]{js_design_patterns}, \cite{crockford_private}. It is also easy to emulate multiple inheritance, traits and mixins.

We can observe that:
\begin{itemize}
    \item The constructor takes as an argument a \texttt{specification} object [0]. This object contains the properties that define the initial state of the object being created.
    \item This object is then deconstructed [1] to extract those properties and 
    \item A \texttt{method} 
\end{itemize}



