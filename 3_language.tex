\chapter{Dual programming language}\label{chap:lang}

\section{Introduction}
% perhaps different (earlier) chapter -- background? philosophical?
I recognize that the textual representation is very useful and will be used by programmers if available.


First prototype

iterations

background: BNF? just mention?

Advantages of a textual representation:
Find and replace
Block editing
Flexibility
Quick editing

In order to make a visual language a viable choice for textual-oriented programmer it's important that a language has a solid textual representation.

This chapter describes the textual representation of Dual, which is the basis for the executable representation for the interpreter (the syntax tree) and for any other editable representation -- including the block-based visual representation implemented in the prototype.

While implementing this project I learned that programming language design is a tremendous task, especially if the language being designed is intended to be of real-world use. Designing and implementing such a language absolutely from scratch, while introducing useful innovation cannot be done within the time limits of research for a thesis, unless perhaps by an experienced language designer. But such experience has to be gained somehow and this is an excellent opportunity.

The character of this research project is exploratory, although I intend to further develop ideas described here and continue my research, which will, as I hope, eventually result in creation of an innovative and useful language ready for real-world use.

That aside, I believe that at least some of the ideas described here are -- in a varying degree -- innovative and worth exploring further.

The evolution of programming languages is a gradual process. And so is the process of designing a single language. The approach that I found effective was iterative refinement, addition, testing, and sometimes subtraction of features. In practice this translates to intermediate designs and implementations being rearranged into new forms, with some discarded. I did not arrive at something that I could call the final form of the language, so a lot of the features described here are subject to change and improvement. This might show in descriptions, where along with talking about the prototype implementation I propose alternatives and refinements.

In chapter [[]] I discuss features and ideas that reached only the design stage and were not implemented, although some of them will be further researched outside of scope of this thesis.

Most of the features of the language and editor in the prototype are implemented as a proof-of-concept, although some are more refined than others in order to fulfill the major goals of this thesis, one of which was to implement a working non-trivial application in the language. I cover a lot of design and implementation surface, only delving deep into some features that are relevant to core ideas that I wanted to convey in this thesis. The other features are implemented necessarily, as steps along the path to practical application of those ideas.

For these reasons the created environment is by no means complete and ready for use in developing complex applications. The degree of completeness of the project is reflected in:
\begin{itemize}
    \item The core language, which is sufficiently expressive to implement any algorithm, i.e. Turing-complete in the practical sense\footnote{That is, it is as Turing-complete as JavaScript or C. No existing language is really Turing-complete in the absolute sense, because of physical hardware limitations.}. A simple Brainfuck\footnote{Which is easily demostrable to be Turing-complete: \url{http://www.iwriteiam.nl/Ha_bf_Turing.html} and thus} interpreter is included as an example program [[resources]] to demonstrate this
    \item Implementation of several interesting additional features of the core language, which are described in this chapter
    \item The ablility of the language to implement also non-trivial applications. This is demonstrated by implementing a clone of Pac-Man, described in Chapter \ref{chap:case}
    \item The direct correspondence of the visual and text representations, with the possibility of parallel dynamic editing; albeit the visual editing part of the editor includes only basic features and is not optimized in terms of performance; details are described in Chapter \ref{chap:editor}
    \item Implementation of the prototype of the language's development environment on top of the web platform, which includes a server-side and a client-side part. It runs locally on the user's machine, but is designed to be easily deployed as an online web application
    \item The editor has several useful features, such as basic support for text editing built on top of the CodeMirror JavaScript component with custom syntax highlighting and integration with the editor. The text editing component is integrated with the visual editing component, so that navigation or changes to each representation are tracked and visible to the user [[]]
    \item [[]]
\end{itemize}

The Pac-Man clone is implemented in a stable subset of the language [[]]
% perhaps list which primitives belong to this subset?

The language was not originally intended as a Lisp-like language of clone thereof, but throughout the research I ended up reinventing some language constructs characteristic of Lisp and learning a lot of and about the language.

An somewhat philosophical interpretation of this would be that Lisp is built on some fundamental principles that are (re)discoverable rather than invented.

% my experience with lisp
% perhaps a natural tendency
% BUT we may observe:

% fix this:
It seems that if we reduce a design of a programming language design to its essentials, we are left with Lisp. Such essentials are:
\begin{itemize}
    \item Syntax. Lisp has a very minimal syntax that is built on the principle of %%
    Any scope is marked with a beginning and an end symbol. Operators are recognized by their position in the source. If we exclude various extensions, flavors and modifications, there's no extraneous characters or symbols
    \item Definitions, declarations
    \item Eval \& apply
\end{itemize}

Even though the language presented in this thesis is complete in the sense of being able to implement any algorithm and non-trivial applications, as exemplified by the Pac-Man clone, it is by no means a complete design. It should be viewed as a snapshot from a continuous design process that is intended to progress in the future.

Provided brief specification and description primarily describes the implementation provided with this thesis. But it is also annotated with suggestions for improvements or, in other words, descriptions of the more refined, future design, which itself is subject to change.

\section{Name}
The name of the language reflects the original core concept of the two concurrent representations: the visual and text. Throughout this research however, I was able [[]] to generalize the concept to any number of possible representations.

But duality is present in many aspects of the language [[see definition/evaluation duality, chap:design]] and in a more general, philosophical sense it's a fundamental principle of reality. The language is something of an exploration of this and other fundamental principles, if not of reality, then at least of programming language design. So, since ``Basic'' is already taken and ``Fundamental'' sounds somewhat pejorative, I stuck with the original name. 
:D

\section{Grammar and syntactic features}
% introduce features as needed
Among the main design goals for the prototype of the language were simplicity and clarity. I wanted a language that is easy to parse and transform to a different representation. This restriction suggests that the syntax should be as minimal as possible. A language with one of the most (if not \textit{the} most) minimal syntax is Lisp\cite{syntaxation}.

An interpreter for List is also trivial to implement, so this is a good starting point.

There are many approaches to implementing interpreters for LISP in JavaScript\footnote{\url{http://ceaude.twoticketsplease.de/js-lisps.html}}, but the general principles are the same.

Although I opted for a minimal syntax, I did not want it to be exactly Lisp-like, as I thought the syntax of this language could be considerably improved -- in terms of ease of use and parsing by a human -- with a few simple adjustments, without significantly increasing the complexity of the interpreter.

There primary criticisms of Lisp's syntax are:
\begin{itemize}
    \item Almost absolute uniformity of syntax makes the source code difficult to read by a human and thus [[]]
    \item In general it is hard to teach\cite{wadler_critique}, because complex code gets easily confusing
    \item The more nested the syntax tree, the harder it is to keep track of and balance parentheses; there tends to be a lot of closing parentheses next to each other in the source\footnote{\url{http://c2.com/cgi/wiki?LostInaSeaofParentheses}}
\end{itemize}

\subsection{Basic syntax}
% note: the footnote should be a citation
Before the primary notation of Lisp, namely S-expressions\footnote{\url{http://www-formal.stanford.edu/jmc/recursive/node3.html}}, was established -- a slightly different and, in my opinion, slightly more readable notation was used in the early theoretical publications about the language, called meta-expressions or M-expressions\cite{mexpr}. I first simplified this notation in the following way:
\begin{itemize}
    \item I dropped the semicolon \texttt{;} as a separator for arguments, as it is entirely superfluous and there is no need for the programmer to type it or the parser to be concerned with it; this means that the only separating characters are the whitespace characters, exactly as in pure S-expressions
    \item The primary bracketing characters are square brackets (\texttt{[]}) instead of parentheses; the reason for that design choice is that these are easier to type than parentheses or curly brackets (as they do not require holding the shift key), which matters considering the ubiquity of these characters in the source code
    \item Expression's operator name is written before the opening bracket that precedes the list of arguments, as in \texttt{operator[argument-1 argument-2 ... argument-n]}
    \item I decided not to include any other syntax in the first prototype, as the one described so far is entirely sufficient to represent any Lisp-like expressions -- it maps directly to S-expressions\footnote{Any additions can be considered syntax sugar. Such syntax extensions were introduced in later prototypes and designs and some of them are included in the final prototype included with this thesis; see [][]} 
\end{itemize}

This gives a notation that is somewhat in between S-expressions and M-expressions. This was the basic syntax of the first prototype of the language. It can be defined\footnote{This definition is included here only for the sake of formality. I believe that for such a simple grammar BNF seems to introduce more noise and is unnecessarily more complex than a textual description in terms of regular expressions or simply verbatim parser source code. For these reasons any extensions to this basic grammar will later on be described in these ways.}, using pure, left-recursion-free BNF notation with the addition of a regular expression (between \texttt{/} delimiters) in the definition of \texttt{<word>}, as follows:
\begin{lstlisting}
    <expression>     ::= <word> | <call>
    <call>           ::= <operator> <argument-list>
    <operator>       ::= <word> <argument-lists>
    <argument-list>  ::= "[" <arguments> "]"
    <word>           ::= /[^\s\[\]]+/
    <argument-lists> ::= <argument-list> <argument-lists>
                         | ""
    <arguments>      ::= <expression> <arguments> | ""
\end{lstlisting}

The regular expression can be read as ``any character which is not whitespace, \texttt{[} or \texttt{]}''. This means that aside from whitespace, which acts as expression separator there are only two special characters -- the square brackets -- that the parser have to worry about. For this reason it is very trivial to implement.

The above -- somewhat verbose -- definition is obviously somewhat similar to a Lisp BNF description\footnote{\url{http://www.cs.cmu.edu/Groups//AI/util/lang/lisp/doc/notes/lisp_bnf.txt}, \url{http://cui.unige.ch/db-research/Enseignement/analyseinfo/LISP/BNFlisp.html}}.

An example of a valid expression in light of this definition would be:
\begin{lstlisting}
    do [
        bind [a 3]
        bind [b 5]
        bind [is-a-greater if [>[a b] true false]]
        is-a-greater
    ]
\end{lstlisting}

Where\footnote{Note: I will introduce brief definitions of language constructs as they appear in the presented listings. For a comprehensive list of primitives and functions see Section \ref{sec:primitives} of this chapter.}:
% shorten this and move details to the end

\begin{itemize}
    \item \texttt{do} is a language primitive that serves the role of a single block of code, much like blocks delimited by \texttt{\{} and \texttt{\}} in C-like languages.
    \item \texttt{bind} is a basic construct for defining variables, like \texttt{var} or \texttt{define} in other languages.
    \item \texttt{if} serves as a basic conditional evaluation construct. Its semantics are like those of the analogous construct in Lisp.
\end{itemize}

Advantages of this M-expression-based notation over S-expressions are:
\begin{itemize}
    % perhaps the remark about C-like syntax should be moved to Background and extended
    \item Easier to parse by a human. Operators are clearer distinguished from operands. This is arguably because this notation is more familiar, bearing a similarity to the general mathematical notation (as in \texttt{f(x)}) and the most popular programming language syntax -- the C-like syntax\footnote{See \url{http://www.tiobe.com/tiobe_index}; 11 out of the top 20 languages as of June 2016 have C-based syntax (by this classification: \url{https://en.wikipedia.org/wiki/List_of_C-family_programming_languages}). If we extend the syntax family to Algol-like, its virtually 20 out of 20 -- \url{http://www.martinrinehart.com/pages/genealogy-programming-languages.html}. There are no languages with Lisp-based syntax among the most popular ones.}
    \item If an expression has another expression as its operator, it is written as \texttt{op[args-1][args-2]}, which reduces the amount of nesting and thus the amount of bracketing characters appearing next to each other in the source code. Compare the equivalent S-expression: \texttt{((op args-1) args-2)}; and with multiple levels: \texttt{op[args-1][args-2][args-3][args-4]} vs \texttt{((((op args-1) args-2) args-3) args-4)}
\end{itemize}

An interesting property of this syntax, that, depending on the context could be classified as advantage, disadvantage or neither is that the sequence of characters \texttt{[[} is not legal, whereas in Lisp the analogous sequence \texttt{((} is.

% might elaborate on that
% this means that in '[[string]] the inner brackets would have to be escaped
% if this was legal, than any string of text would be a syntactically legal program, provided that [ and ] are balanced
% a naive scheme implementation would have that property
% http://stackoverflow.com/questions/4900342/why-is-the-hyphen-conventional-in-symbol-names-in-lisp arbitrary lisp symbol names |this *@^! symbol is valid - why is that po_ss_ib_le?|

Alas, this simple notation doesn't do away with a lot of other problems inherent in any minimal syntax, because such syntaxes have the property of being very homogenous. In the next subsections and later throughout this thesis I gradually introduce extensions, which make the syntax a little bit more diverse. Keep in mind that every special character that is introduced, is taken away from the set of possible \texttt{<word>}-characters, which implies that the regular expression for \texttt{<word>} is changed accordingly.

\subsection{Comments, whitespace and the syntax tree}\label{sub:comments}
% describe comments
The parser was further extended with support for comment syntax similar to the ones found in Ada, Haskell or Lua:
\begin{lstlisting}
    -- a comment that extends until the end of the line
    -- an expression that computes square root of 81:
    sqrt[81]
    
    --[
        this is a multiline comment
        
        --[
            multiline comments can be nested
            
            as long as [ and ] are balanced,
            anything can be nested within
            multiline comments
            
            for example:
            --[this is a comment that includes
            a piece of code *[7 7],
            which would evaluate to 49]
        ]
    ]
\end{lstlisting}

In principle multiline comments could be implemented simply with the syntax analyzer checking the operator of the expression being parsed, and if it is \texttt{--}, treating such expression as a comment. The fact that this expression was already parsed and transformed into a structural tree-like form could be taken advantage of while generating documentation from comments. For example we could define a following \acrlong{dsl}\footnote{Inspired by \url{https://en.wikipedia.org/wiki/JSDoc})} for documentation:
\begin{lstlisting}
    --[
        -- the below is a documentation comment
        -- followed by the documented piece of code:
        --[[
            Calculates the circumference of the Circle.
            
            override!
            deprecated!
            
            this [circle]
            
            -- The circumference of the circle:
            return [number]
        --]]
        
        define [calculate-circumference procedure [
            mul[2 math.pi this.radius]
        ]]
    ]
\end{lstlisting}

Nevertheless the implementation in the prototype treats the comments as a stream of characters, taking into account nesting and balancing of brackets, but doesn't keep them as a tree-like structure.

% describe
Whitespace characters and comments have no semantic significance, unless serving as separators could be considered one. After building the syntax tree, bracketing characters also serve no purpose and can be safely be discarded, without influencing the interpretation of the program.

Despite this, all of these are included in the syntax tree generated by the parser. Storing these characters in the syntax tree means that the entirety of the textual representation, in structural form, is accessible by any other representation we might devise. This allows for implementation of variety of interesting ``smart'' language and editor features.

A thing to note at this point is that such syntax tree can't be described as ``abstract''\footnote{According to these definitions: \url{https://en.wikipedia.org/wiki/Abstract_syntax_tree}, \url{http://c2.com/cgi/wiki?AbstractSyntaxTree}}, as it includes all of the ``concrete'' syntax, with its runtime-insignificant elements. The syntax tree that is built for the Dual language is also later extended with references to other representations of code. For these reasons, I will later on use the acronym \acrlong{est} to refer to the representation used internally by the Dual language interpreter.

This representation greatly simplifies the implementation of and integrates with the language the following features:
\begin{itemize}
    \item Automatic indentation % elaborate
    \item Documentation comments. Comments can easily be associated with corresponding code blocks (syntax tree nodes), which can be useful for automatically generating documentation in any format
    \item Any expression can be unparsed to its original form straight from syntax tree, which can be used for debugging
    \item This also means that any expression can be stringified on-the-fly and this string can be used as a value in the program. This feature allowed me to completely omit definition of strings at the parser level, although this is not a very efficient solution. Nevertheless keeping strings in such structural form -- as syntax trees -- in combination with pattern matching enables language-native structural manipulation of strings\footnote{See: \url{https://reference.wolfram.com/language/tutorial/WorkingWithStringPatterns.html} and \url{https://en.wikipedia.org/wiki/Pattern_matching\#Pattern_matching_and_strings} for similar concepts.}\footnote{I chose the structural representation as the only representation, because the performance penalty is acceptable in the prototype implementation. An obvious and very simple optimization would be to keep the raw form of the string as a value in the corresponding syntax tree node. Having these two forms alongside each other would enable the programmer to use the familiar string manipulation methods as well as structural manipulation.}. For example we could write:
    \begin{lstlisting}
        bind [str '[A quick brown fox jumps over
                    the lazy dog]]
                    
        bind [words [_ _ third-word {rest}] str]
        
        bind [characters [_ _ third-letter {rest}]
              third-word]
        
        -- logs "o" to the console:
        log [third-letter]
    \end{lstlisting}
    
    Where \texttt{words} deconstructs a string into single words and binds these words to identifiers provided as its arguments and \texttt{characters} performs an analogous operation on the single character-level. The notation \texttt{\{rest\}} matches zero or more arguments (see Section \ref{sub:rest} for details). \texttt{log} outputs the values of its arguments to the JavaScript console.
\end{itemize}

Such representation was implemented in order to fulfill the design goal of direct and complete mapping of the textual representation into any other representation.

\subsection{Numbers}
Numbers in the language are represented as JavaScript numbers. This means that there's only one number type -- 64-bit floating point\footnote{Defined by the IEEE 754 standard: \url{http://www.iso.org/iso/iso_catalogue/catalogue_tc/catalogue_detail.htm?csnumber=57469}, \url{http://www.2ality.com/2012/04/number-encoding.html}}. They are implemented as follows:
\begin{itemize}
    \item When a word is tokenized by the parser, it is converted to a JavaScript number with a Number type constructor, which returns either the corresponding value (if the word is parsable to a number) or the value \texttt{NaN}. In the former case, the numerical value is stored in the appropriate syntax tree node, as its \texttt{value} property.
    \item Upon evaluation, a syntax tree node is checked for the \texttt{value} property. If it has one it is given as the result of the evaluation.
    \item The fact that a number is stored as a syntax tree node, which contains the its string representation and its raw value, both obtained from the source code during parsing means that conversion from a number literal to string is zero-cost, which could be useful for optimization.
\end{itemize}

This shows a possible way of optimizing the representation of strings, which I intend to introduce in a future version of the language.

\subsection{Zero and single argument expressions}
In order to reduce the amount of \textit{closing} brackets appearing next to each other in program's text, two additional simple notations were introduced. The first is addition of the pipe special character (\texttt{|}). This character is used for single-argument expressions, as in:
\begin{lstlisting}
    -- compute factorial of 32:
    factorial|32 -- equivalent to factorial[32]
    
    -- find 9th Fibonacci number:
    fibonacci|9 -- <=> fibonacci[9]
    
    -- compute sine of pi
    sin|pi -- <=> sin[pi]
    
    -- compute cosine of the number that is the result of
    -- multiplication of pi and 5!:
    cos|*[pi factorial|5] -- <=> cos[*[pi factorial[5]]]
    
    -- convert 33.2 to an integer (truncate .2):
    to-int|33.2 -- <=> to-int[33.2]
    
    -- construct a list with one item,
    -- which is a string "hello"
    list|'|hello -- <=> list['[hello]]
\end{lstlisting}

The above example shows that if a function is invoked with one argument, we can omit the closing brace and replace the opening brace with \texttt{|}. The parser produces equivalent syntax tree.

Another special character (\texttt{!}) was introduced for analogous use for zero-argument expressions (procedures):
\begin{lstlisting}
    -- invoke a procedure that changes some
    -- state variables in its outer scope:
    set-initial-state! -- <=> set-initial-state[]
    
    -- sum two random numbers:
    -- <=> +[random[] random[]]:
    +[random! random!] 
    
    -- bind a value returned by
    -- an immediately invoked procedure to 
    -- an identifier
    -- <=> bind [forty-two procedure [42][]]
    bind [forty-two procedure [42]!] 
    forty-two -- evaluates to 42
\end{lstlisting}

\subsubsection{In combination with macros}\label{subsub:macros}
A unique macro system, described in Chapter \ref{chap:design}\footnote{This macro system is not included in the final version of the prototype, although a proof-of-concept of it that I implemented in earlier prototypes is the basis for this description.} in combination with these two (\texttt{|} and \texttt{!}) special characters help reduce the amount of bracketing characters even further.

For example, if we define a \texttt{match*} and \texttt{of*} macros as described, the following expression:
\begin{lstlisting}
    bind [x 99]
    
    -- will log "x is greater than one":
    match* [x]
    | of* [<|0] [log|'[x is negative]]
    | of*   [0] [log|'[x is zero]]
    | of*   [1] [log|'[x is one]]
    | log|'[x is greater than one]
\end{lstlisting}

which is somewhat similar syntactically to ML-style\footnote{\url{https://en.wikipedia.org/wiki/Standard_ML\#Algebraic_datatypes_and_pattern_matching}} languages, could be translated into the following:
\begin{lstlisting}
    bind [x 99]
    
    -- will log "x is greater than one":
    apply [
        of [<|0 log|'[x is negative] 
            of [0 log|'[x is zero]
                of [1 log|'[x is one]
                    log|'[x is greater than one]
                ]
            ]
        ]
        x
    ]
\end{lstlisting}

where \texttt{apply} would be defined analogously to Lisp's \texttt{apply}. \texttt{of} would be defined as a function primitive with arity 2..*, which treats its penultimate argument as the function's body and all the preceding arguments as patterns for the function's arguments. The last argument is used when such a function is called and the values supplied as arguments don't match the patterns. If the argument is a function, it will be called with the same values as arguments and if it's a value it will be returned.

I used such a solution for pattern matching in the early prototypes, but replaced it with a native \texttt{match} construct (described in Section \ref{sec:primitives}) for performance reasons. Nevertheless this shows that a few simple, but general syntax rules and a powerful macro system, can be a very flexible tool for extending syntax.

The pattern matching mechanism is explained in the next subsection.

\subsection{Pattern matching}
A simple, yet powerful pattern-matching facility was added to the language.
% this is actually more powerful than in Common Lisp http://nullprogram.com/blog/2013/01/20/\#Destructuring

Pattern matching works with bindings, functions (although the primary function-producing expression in the prototype doesn't use it by default), \texttt{match} primitive and macros (not available in the prototype).

The pattern matching works in a way similar to most other languages that support this feature (e.g. ML family). The general rules are\footnote{For brevity I assume here that `of` is a primitive that works like described in \ref{subsub:macros}, where the alternative argument is optional. This is how it was implemented in an early prototype of the language. If no viable alternative was present, an error was thrown.}:
\begin{itemize}
    \item A literal (strings or numbers are supported) value matches itself:
    \begin{lstlisting}
        -- computes factorial of a number
        bind [factorial
            of [0 1 
            of [n *[n factorial[-[n 1]]]]]
        ]
        
        -- logs `120`:
        log [factorial|5]
    \end{lstlisting}
    \item An identifier (word) matches any value, which is then bound to the identifier:
    \begin{lstlisting}
        bind [simple-print of [x log|x]]
        
        -- logs `3`:
        simple-print[3]
    \end{lstlisting}
    \item A wildcard pattern (\texttt{\_}) matches any value, but doesn't bind:
    \begin{lstlisting}
        -- returns its third argument,
        -- discards the rest:
        bind [get-third of [_ _ x x]]
        
        -- logs `3`:
        log [get-third[1 2 3]]
    \end{lstlisting}
    As such it can be useful for discarding some values, depending on other values or extracting some values from a structure (see next point).
    \item The following expression-patterns are supported:
    \begin{itemize}
        \item \texttt{list} or \texttt{\$} is used to destructure lists:
        \begin{lstlisting}
            bind [$[_ _ third-element] $[0 1 2]]
            
            -- logs `3`
            log [third-element]
            
            -- it works for arbitrarily nested
            -- lists as well
            bind [
                $[  _ $[  _ pick   _   _]   _]
                $['|a $['|b  '|c '|d '|e] '|f]
            ]
            
            -- logs `c`:
            log [pick]
        \end{lstlisting}
        \item Comparison operators (\texttt{= < <= >= <>}) match if a value passes the comparison; it can be viewed as a shorthand notation for simple guards\footnote{\url{https://en.wikibooks.org/wiki/F_Sharp_Programming/Pattern_Matching_Basics\#Using_Guards_within_Patterns}}:
        \begin{lstlisting}
            -- returns the sign of a number
            -- note: `-#` is the unary `-` operator:
            bind [sign of [=|0    0
                       of [<|0 -#|1
                       of [>|0    1]]]
            ]
            
            -- logs `-1`:
            log [sign|-77]
        \end{lstlisting}
        \item % rest parameters
        \item Other pattern-expressions are not supported and using them will result in a mismatch.
    \end{itemize}
\end{itemize}

% ???
The above examples show pattern matching used for destructuring values and binding their components to identifiers and for function definitions. There's also a \texttt{match} primitive, which can serve the role of a \texttt{switch} statement from C-like languages. Although pattern matching makes it much more powerful than that, as any values supported by the pattern matching system can be matched, including lists, which allow us to switch on multiple values and in any combinations.

The \texttt{match} primitive's first argument is a value to match and all subsequent arguments are two-element lists, where the firs element is the pattern to match and the second is the expression to evaluate in case of a match. The primitive tries the matches in order and only evaluates the expression, related to the successful match, which is the first one that matches. The subsequent matches are not evaluated.

\begin{lstlisting}
    bind [state '|game-on]
    
    -- will execute the `play` procedure:
    match [state
        $['|game-on play!]
        $['|game-paused display-pause-menu!]
        $['|game-screenshot capture-screenshot!]
    ]
    
    -- ...

    -- note: . is the access operator
    -- .[a b c] is equivalent to a.b.c in other languages
    bind [$[x y] .[player postion]]
    
    -- we can easily replace complex conditions:
    match [$[x x y y]
        $[
            $[>|0 <|screen-width >|0 <|screen-height]
            log|'[player visible]
        ]
        $[_ log|'[player not visible]]
    ]
    
\end{lstlisting}
\begin{itemize}
    \item 
\end{itemize}

With an arsenal of these few simple pattern matching tools we can use a lot of useful features, which further add expressivity to the language. We can also imagine many possible extensions and generalizations, as briefly discussed in Chapter \ref{chap:design}.

% ??? discard?:
\begin{itemize}
    \item Destructuring assignments or, more precisely, destructuring definitions.\footnote{Destructuring could easily be extended to mutation as well, although I have found it sufficient to be usable only in definitions, while implementing the prototype.}. An example of such definition would be:
    \begin{lstlisting}
        bind [$[a b $[c d]] $[1 2 $[3 4]]
        bind [
            $[  _   x   y  {  rest  }]
            $['|a '|b '|c '|d '|e '|f]
        ]
        
        -- logs `1 2 3 4`:
        log [a b c d]
        
        -- logs `b c ["d", "e", "f"]`:
        log [x y rest]
    \end{lstlisting}
\end{itemize}

% match

\subsection{Rest parameters and spread operator}\label{sub:rest}
% quotation, substitution
Another syntax extension that I introduced involved two additional special bracketing characters: \texttt{\{} and \texttt{\}}, which serve several purposes:
\begin{itemize}
    \item Rest parameters mechanism known from Lisp\footnote{\url{https://www.gnu.org/software/emacs/manual/html_node/elisp/Argument-List.html}}, recently also adopted in JavaScript (as of the ECMAScript2015 standard\footnote{\url{https://developer.mozilla.org/pl/docs/Web/JavaScript/Reference/Functions/rest_parameters}}). That is, for example:
    \begin{lstlisting}
        bind [variadic-function of [a b {args}
            log [a b args]
        ]]
        
        -- logs `1 2 [3, 4, 5, 6]`:
        variadic-function[1 2 3 4 5 6]
    \end{lstlisting}
    This enables the user to easily define variadic functions, which can be called with a variable number of arguments. 
    % FIXME, clarify or throw away
    %If the name of last parameter of the function is between curly braces, then this name will be associated with a (possibly empty) list, which holds the first value at the position of the rest parameter and all subsequent values
    This works in any place, where pattern matching works:
    \begin{lstlisting}
        bind [$[a b {rest}] $['|a '|b '|c '|d '|e]]
        
        -- logs `["c", "d", "e"]`
        log [rest]
    \end{lstlisting}
    
    thus enabling non-exact matching.
    
    \item Spread operator (also inspired by the analogous feature from ECMAScript2015):
    \begin{lstlisting}
        bind [f of [a b c d e f log [a b c d e f]]]
        bind [args $[8 7 6]]
        
        -- logs `9 8 7 6 5 4`:
        f[9 {args} {$[5 4]}]
    \end{lstlisting}
    This provides a much nicer and more powerful alternative to \texttt{apply}, Lisp's fundamental function, which applies a function to a list of arguments. It's a way to flatten any list onto a list of arguments. This works for multiple values and lists as well:
    \begin{lstlisting}
        -- alternative way to achieve the same
        -- result as in the previous listing
        -- logs `9 8 7 6 5 4`:
        f[{9 args $[5 4]}]
    \end{lstlisting}
    \item String interpolation notation:
    \begin{lstlisting}
        bind [name '|Bill]
        
        -- logs `Hello, Bill.`
        log ['[Hello, {name}.]]
    \end{lstlisting}
    
    As we can see this gives us a very convenient notation for string interpolation, similar to e.g. template literals in JavaScript\footnote{\url{https://developer.mozilla.org/en-US/docs/Web/JavaScript/Reference/Template_literals}}.
    In order to escape curly braces, they should be doubled:
    \begin{lstlisting}
        -- logs `Hello, {name}.`
        log ['[Hello, {{name}}.]]
    \end{lstlisting}
    
    I also added a special type of string -- an HTML string, where interpolation notation is the other way around -- double braces cause substitution, single braces do nothing:
    \begin{lstlisting}
        bind [name '|Bill]
        -- logs `<h1>Hello, Bill.</h1>`
        log [html'[<h1>Hello, {{name}}.</h1>]]
        
        -- logs `<h1>Hello, {name}.</h1>`
        log [html'[<h1>Hello, {name}.</h1>]]
    \end{lstlisting}
    
    This is to enable embedding CSS and JavaScript code inside those strings, without having to constantly escape brace characters.
    
    \item Unquote notation for macros\footnote{Analogous to \texttt{unquote} or \texttt{,} in Lisp: \url{https://docs.racket-lang.org/reference/reader.html\#\%28part._parse-quote\%29}} -- see Chapter \ref{chap:design}.
\end{itemize}

on duality of binding and evaluation
expand in chapter 6 \ref{chap:desgin}
    optional parameters

% http://nullprogram.com/blog/2013/01/20/
% pattern matching gives a simple way to define optional parameters and give them default values
% http://nullprogram.com/blog/2013/01/20/\#clojure-parameters
% Common Lisp also has destructuring
% Clojure, ClojureScript -- popular Lisp-like
% Crockford: mistakes in C-syntax; rise of langs w/ different syntax (CoffeeScript)

%%%
Strings can be represented structurally, as syntax trees. This, in combination with pattern matching allows for 

\cite{eloquentjs}
Between S and M expressions 

    introduce it, but state that BNF ain't good and that it's not the best way of looking at such a simple language, only complicates reasoning

 \section{Basic primitives and functions}\label{sec:primitives}
 
 {}
 rest parameters and spread
 
 log
 
 list/\$
Below is a description of the primitives and functions supported by the Dual language. Each item is structured as follows:
\begin{itemize}
    \item \texttt{<name> [<arguments>]}
    
    <description>
    
    Where \texttt{<name>} is the name of the function/primitive and \texttt{<arguments>} are either the names that describe the arguments of the function/primitive or its arity. That is, the number of arguments that the function/primitive is defined for. This can be a fixed value (e.g \texttt{1}), a fixed range of values (e.g. \texttt{0..3}) or a range of values without an upper bound (e.g. \texttt{0..*}, which means 0 or more).
    
    <description> is a brief description of the function/primitive.
\end{itemize}

\subsection{Language primitives}
The Dual language supports the following primitives:
\begin{itemize}
    \item \texttt{do [0..*]}
    
    Evaluates its arguments in order and returns the value of the last argument.
    
    \item \texttt{bind [name value]}
    
    Evaluates its second argument and binds this value to the name of the first argument. This name is bound within the current scope. This is a basic construct for defining variables, like \texttt{var} or \texttt{define} in other languages. Significant semantics here are that new scopes are introduced by function bodies, macro bodies and match expression bodies. The primitive also supports pattern matching to deconstruct the value and bind its components to possibly several variables. In that regard it works a lot like JavaScript's destructuring assignment\footnote{\url{https://developer.mozilla.org/pl/docs/Web/JavaScript/Reference/Operators/Destructuring_assignment}} or similar features in other languages, such as Perl or Python. This primitive can be used only for binding names that don't exist in the scope at the point of its invocation. There are other constructs for mutating and modifying existing variables. There is no hoisting\footnote{\url{https://developer.mozilla.org/en-US/docs/Web/JavaScript/Reference/Statements/var\#var_hoisting}}, as definitions are processed in order in which they appear in code.
    
    \item \texttt{if [condition consequent alternative]}
    
    This primitive serves as a basic conditional evaluation construct. Its semantics are like those of the analogous construct in Lisp. It accepts 3 arguments: first the \texttt{condition} expression, then the \texttt{consequent}, that is, the expression to be evaluated if the value of the condition is \textit{not false} (note that this is a strict rule; any other value than \texttt{false} is interpreted as \texttt{true}; every conditional construct in the language follows this rule). The third argument, the \texttt{alternative} is the expression that is evaluated otherwise.
    
    \item \texttt{while [condition body]}
    
    A basic loop construct. If \texttt{condition} is equivalent to \textit{not false}, evaluates \texttt{body}. Repeats these steps until \texttt{condition} evaluates to \texttt{false}. Returns the value of the last evaluation of \texttt{body} or \texttt{false} if the body was not evaluated.
    
    \item \texttt{mutate* [name value]}
    
    If a variable identified by \texttt{name} is defined within the current scope or any outer scope, changes (mutates) its value, so it now refers to the result of evaluating the \texttt{value} argument. The scopes are searched from the innermost to the outermost, in order. If the \texttt{name} argument doesn't identify any variable, an error is thrown. Returns the scope (environment), in which the primitive was evaluated. [[first-class environments, Bla paper?]]
    
    \item assign
    
    \item code
    
    \item macro
    
    \item of
    
    \item of-p
    
    \item procedure
    
    \item match
    
    \item cons
    
    \item invoke*
    
    \item .

    \item :
    
    \item @
    
    \item dict*
    
    \item async*
\end{itemize}

Basic functions and values:

[[]] Mapping to a JavaScript equivalent

\begin{itemize}
    \item \texttt{true} and \texttt{false} evaluate to their respective boolean values. \texttt{\_} is an alias for \texttt{true} when used outside of pattern-matching. This enables a convenient compatibility between \texttt{match} and \texttt{cond}: if we're matching a single value and want to have a default case, then \texttt{\_} is used to match any value. Similarly, if \texttt{\_} is given as a condition in the last alternative of \texttt{cond}, it will evaluate to \texttt{true} and work as the default case.
    \item \texttt{undefined} evaluates to JavaScript's undefined[[link]].
    \item \texttt{typeof} wraps JavaScript's \texttt{typeof} operator.
    \item \texttt{or} and \texttt{and} are the basic logical operators -- analogous to \texttt{||} and \texttt{\&\&} in JavaScript.
    \item \texttt{any} and \texttt{all} are like the above, but accept variable number of arguments. These return either \texttt{true} or \texttt{false}.
    \item \texttt{not} is the negation operator (\texttt{!})
\end{itemize}

see text files ideas

language's syntax/grammar
    language's grammar in BNF-notation, without left-recursion, a bit verbose:
    <expression>     ::= <word> | <call>
    <call>           ::= <operator> <argument-list>
    <operator>       ::= <word> <argument-lists>
    <argument-list>  ::= "[" <arguments> "]"
    <word>           ::= /^[^\s\[\]]+/
    <argument-lists> ::= <argument-list> <argument-lists> | ""
    <arguments>      ::= <expression> <arguments> | ""

    less verbose (not sure if correct):
    <expression>       ::= <word> <expression-lists>
    <word>           ::= /^[^\s\[\]]+/
    <expression-lists> ::= "[" <expressions> "]" <expression-lists> | ""
    <expressions>      ::= <expression> <expressions> | ""

    whitespace is mostly non-significant;
    it doesn't influence the semantics of a program, although it is stored in the syntax tree and can be accessed by special primitives, such as `string`

language's semantics
    AST is a tree of nested expression objects
    there are two types of expressions:
        word
            an identifier/name
            usually refers to some value in the accessible scope
            could also be treated as literal string (possibly parsed to a number), depending on the operator that evaluates it
        call
            an application
            consists of an operator (which is an expression that is being applied) and arguments (which are fed to the operator)
            should produce a value

    every language element is an expression and has a value

    define
        aliases: :
        arity: 2
        arguments: name, value
        description:
            creates a value accessible in the closest scope under the `name` label; the value is the result of evaluating `value`
        returns: the result of evaluating `value`

    sequence
        aliases: seq, do
        arity: * (0..infinity)
        description: evaluates its arguments in order
        return: the value of its last argument

    string
        aliases: $
        arity: *
        description:
            returns its arguments' names and the whitespace between them as a string value, as in:
                $[Hello, wolrd!]
            would be more-or-less equivalent to Java string:
                "Hello, world!"
        notes:
            this kind of string is multiline and could work with embedded expressions, as in:
                $[Hello, \$[name]!]
            if `name` would refer to another string of value e.g. "Alan" this would produce a string like:
                "Hello, Alan!"

    if
        aliases: conditional
        arity: 3
        arguments: condition, then-expression, else-expression
        returns:
            if `condition` evaluates to `true`, the value of `then-expression`, otherwise the value of `else-expression`
            alternatively the condition could be checked against a non-zero value instead of `true`
        note:
            I will most likely drop the concept of booleans entirely and treat 0 as a special value, representing `false`, `null`, `undefined`, empty
            any non-zero value would then represent `true`

    while
        arity: 2
        arguments: condition, loop-body
        description: a basic loop construct; if `condition` is equivalent to `true`, evaluates `loop-body`; repeats these steps until `condition` evaluates to `false` (0)
        returns: `false` (0)

    number
        aliases: \#
        arity: 1
        description:
            parses its argument's name as a number, as in:
                \#[3]
            evaluates to the number 3
        returns: a number parsed from its argument's name

    boolean
        aliases: ?
        arity: 1
        description:
            parses its argument's name as a boolean, as in:
                ?[true]
            evaluates to `true`
        returns: `true` or `false`, depending on its argument name
        note: likely will be dropped from the language

    print$
        arity: *
        description:
            works like `string`, but also prints its value to the JavaScript console


    todo:
        greater-than
        aliases: >, gt
        arity: 2
        arguments: a, b
        description: checks if `a` is greater than `b`
        returns:
            `true` if `a` greater than `b`, `false` otherwise
            alternatively `a` if `a` greater than `b` and non-zero, infinity if `a` equal to zero, but greater than `b` and zero otherwise

        less-than
        aliases: <, lt
        description: analogous to greater-than, only checks if `a` less than `b`

        ...other operators and common functions

        function
            aliases: fun, /
            arity: 1..*
            arguments: arg* (zero or more), function-body
            description:
                creates a function value
                all the arguments up to the last are the names of the function arguments (must be identifiers); the last argument is the `function-body`
            returns:
                a function that accepts the defined number of arguments and evaluates them in the context of the `function-body`;
                such a function creates its own local environment on top of the environment it is defined in,
                puts the argumens it was called with in this environment, under the defined labels and evaluates the `function-body` in this environment;
                the return value is the value of the `function-body`

        // branch
        // switch
        // array
        // set

    special elements (not implemeted):
        \
        description: interpreter switch; precedes an interpreter command (which is a valid language identifier), which changes (possibly temporarily) the behaviour of the interpreter
        for example:
            if\3 flag then-do-this else-do-this
        would be equivalent to:
            if[flag then-do-this else-do-this]
       here \ executes a numerical command (3) that makes the interpreter treat next 3 expressions as arguments to the preceding expression, which would be applied to them
       this would be a simple way of getting rid of excess closing brackets ']'
       I have a few other ideas to utilize the interpreter switch, such as comments/documentation.

\section{Basic functions}


\section{Memory model}

language's "memory model"
    to call it a memory model is a stretch;
    every program is evaluated in an environment, which is a JavaScript object (basically a string key-any value map);
    the keys in the environment object are variable names/identifiers;
    the values are the values associated with the corresponding identifiers;
    the environment that is accessible at all times is the root environment, where all globally-accessible values live
    it should be populated with basic arithmetic and logic operators, functions and constructs that are not special language primitives (such as if, while, string, etc.)
    all functions defined in the language have their local environment, build on top of the environment that contains the function definition (closures work)
    functions have access to values in their local environments and all containing environments
    the local environment can have identifiers defined that override outer enivronments' identifiers (but they don't overwrite them)
    this mechanism is built on top of JavaScript's prototype-based inheritance
    
Since it is a thin wrapper over JavaScript it shares its memory model.
Event loop
Single-threadedness


\section{Syntax extensions}
rest parameters
spread operator
built-in template strings, quotation mechanism
html strings 