\chapter{Introduction}\label{chap:intro}

\section{Title}
The term ``Pac-Man-complete'' in the title refers to a somewhat humorous description of the Idris programming language\footnote{\url{http://www.idris-lang.org/}} attributed to the language's author, Edwin Brady\footnote{\url{https://twitter.com/folone/status/494017847585415168}}. In the context of this thesis it means that the designed programming language provides enough features to allow one to write a clone of the classic Pac-Man game in it.

\section{Scope}
This research explores various topics in the field of programming language design. Particularly, the possibilities of integrating and combining multiple representations of the same programming language in a dynamic way. I try to find ways of combining features of visual languages with text-based languages by creating a development environment which lets the programmer work with both representations in parallel or intertwine them in any way.

The created language is used to implement a Pac-Man clone. This provides a demonstration of Dual's capabilities and is a reference for assessing the performance of the implementation.

I discuss further possible improvements in the language's design and performance, as well as set down directions for any future research, which I intend to take on. The exploratory nature of this work lets me cover a fairly broad area, connecting programming language design, computer game development as well as web technologies and web application development.

\section{Choice of subject}
The choice of this particular subject stems from my deep personal interest in programming language design. This research is an opportunity for me to create a project that demonstrates various ideas in this area that I developed over time and to explore and refine them further.

\section{Related work}
Visual languages are not especially popular compared to text-based languages. But recently they have been gaining more popularity, particularly in game development. Chief example and the main cause of this is Unreal Engine, the highly popular and mainstream\footnote{\url{https://en.wikipedia.org/wiki/List_of_Unreal_Engine_games}, \url{http://www.guinnessworldrecords.com/world-records/most-successful-game-engine}} game engine, which in the latest version introduced a visual programming language\footnote{\url{https://docs.unrealengine.com/latest/INT/Engine/Blueprints/}} as its primary scripting language. In fact this is the only scripting language that the engine supports, having dropped the UnrealScript language\footnote{\url{http://www.gamasutra.com/view/news/213647/Epics_Tim_Sweeney_lays_out_the_case_for_Unreal_Engine_4.php}} included in the previous versions. 

With this research, I intend to further the growth of visual programming languages, propose an even more accessible solution and explore possible improvements over comparable technologies.

\section{Goals}
In line with the above, the purpose of this work is to introduce innovation as well as show a practical application of the developed solution. The concrete goals are:
\begin{itemize}
	\item To explore and establish directions where innovation is possible in programming language design and implementation.
	\item To design and implement a programming language, whose \textit{complete} textual representation must be directly and dynamically mappable to a visual representation and vice versa.
	\item To create a prototype of the development environment for the language on top of the web platform.
	\item To evaluate the usability and performance of the system by implementing a clone of Pac-Man and examining the process as well as the results.
	\item To explore and discuss further refinements and possible future design directions.
	\item All throughout this, the designs and implementations shall be compared and contrased with existing solutions.
\end{itemize}

\section{Structure}
This thesis is structured as follows:

Chapter \ref{chap:intro} is this introduction.

Chapter \ref{chap:background} briefly describes technologies and tools used in developing any software described here as well as discusses the essential elements of the theoretical framework upon which the language was built.

Chapter \ref{chap:lang} describes the Dual language: its syntax, semantics and basic design.

Chapter \ref{chap:editor} talks about the architecture, design and serves as a practical documentation of the language's development environment. A comparison to existing visual programming languages is included.

Chapter \ref{chap:case} describes a more-than-trivial application developed with Dual: a Pac-Man clone. Performance of the implementation is assessed and possible adjustments and improvements are discussed.

Chapter \ref{chap:design} elaborates on programming language design both in context of Dual and in general.

Chapter \ref{chap:summary} summarizes and concludes.
